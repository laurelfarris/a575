\documentclass{article}
\usepackage{graphicx}
%\usepackage{natbib}   

\begin{document}

\title{Paper Summary}
\author{Laurel Farris}

\maketitle
A discussion of observations of emission from NGC 253, a starburst galaxy
about 3.44 Mpc away, is presented \cite{alma}. Evidence of star formation 
can be detected with both thermal emission and the H40$\alpha$ emission line.
At the millimeter wavelengths oberved here, both star formation and possibly 
electron temperatures can be constrained.

%\begin{figure}
%    \centering
%    \includegraphics[width=3.0in]{myfigure}
%    \caption{Simulation Results}
%    \label{simulationfigure}
%\end{figure}

%\bibliographystyle{te}
%\bibliography{research}

\begin{thebibliography}{1}
%\addtolength{leftmargin}{0.2in}
%\setlength{\itemindent}{-0.2in}
\bibitem[1]{alma} Bendo, G. J.,Beswick, R. J.,Cruze, M. J.,
                Dickinson, C. , Fuller, G. A. , Muxlow, T. W. B.,
	%"ALMA observations of 99 GHz free-free and H40\alpha line
	"ALMA observations of 99 GHz free-free and H40$\alpha$ line
		emission from star formation in the centre of NGC 253",
	\emph{Royal Astronomical Society}, pp. L80-L84, 2015
\end{thebibliography}

\end{document}
