\documentclass{article}
\usepackage[margin=1.25in]{geometry}
\usepackage{graphicx}
%\usepackage{enumerate}
\usepackage{enumitem}
\usepackage{amsmath}
\usepackage{mathcomp}
\usepackage{color}
\usepackage{mdwlist}
\usepackage{setspace}
\usepackage{hyperref}

\definecolor{myGreen}{rgb}{0.12, 0.3, 0.17}

\definecolor{cobalt}{rgb}{0.0, 0.28, 0.67}
\hypersetup{colorlinks=true, urlcolor=cobalt,}
\urlstyle{same}

\setlength{\parindent}{0em}
\setlength{\parskip}{0.5em}

\setlist{%
    itemsep=-1ex, topsep=0pt}

\title{ASTR 575}
\author{Me}
\date{Fall 2015}

\definecolor{hl}{rgb}{0.61, 0.87, 1.0}
\newcommand{\test}[1]{%
    \begin{center}
        \colorbox{hl}{\parbox{0.9\textwidth}{\emph{#1}}}
    \end{center}}

\newenvironment{wideverbatim}%
{\vskip\baselineskip\VerbatimEnvironment
\begin{Sbox}\begin{BVerbatim}}
{\end{BVerbatim}%
\end{Sbox}\noindent\centerline{\TheSbox}\vskip\baselineskip}
%----------------------------------------------------------------------%
\begin{document}

\maketitle
\addtocontents{toc}{\protect\setstretch{0.1}}
\tableofcontents

\renewcommand{\descriptionlabel}[1]{%
    \ttfamily{#1}}
\setlist[description]{%
    labelsep=1em,
    labelwidth=5em,
    %leftmargin=*
}

% 1
\section{Course Overview}
Uquity of computing in field, and also as skill for other fields.

Philosophy: need to have understanding and flexibility, computing
skills are lifelong learning skills. Note that the way that scientists
have tended to program may be rather different from how computer
programmers in other sectors work!

Operational details: course material (including language discussion),
course resources, homework, notes and note-taking

575 and 535

\test{git branches, how to fork and send pull requests. Learn more
about git command in general.}


% 2
\section{Introduction to computing hardware and NMSU/Astronomy computing}

\subsection{Hardware}
Computer components and functions:
\begin{itemize}
    \item CPU(s): characterized by the rate at which they can do
        operations (\textit{flops} is a standard, to be distinguished from
        \textit{clock speed} in Hz/MHz/GHz).
    \item memory: volatile storage (but relatively fast),
        typically Gbytes per machine.
    \item disk(s)
        \begin{description}%[style=sameline]
            \item [internal] permament storage, typically TBy per disk
                (example: \href{https://en.wikipedia.org/wiki/Serial_ATA}
                {Serial AT Attachment (SATA)}).
            \item [external] USB, eSATA, Firewire
        \end{description}
    \item network: communication between machines
    \item monitor/keyboard
    \item power
    \item bus
\end{itemize}

\test{Understand and be able to articulate the different parts of computer
hardware and their basic functions.}

\subsection{Bits, bytes, and whatever}
\begin{itemize}
    \item Bit: single binary unit of information (0/1).
    \item Byte: 8-bits, usually the smallest unit used to represent
        something. How many different quantities can a byte represent?
\end{itemize}
Data types/sizes:
\begin{itemize}
    \item character (ASCII 1 byte, or more for other character encodings)
    \item short integer (2 bytes)
    \item integer (4 bytes)
    \item long integer (4/8 bytes)
    \item float (4/8 bytes)
    \item double (8/16 bytes)
\end{itemize}

How large a range of data values in a given data type?

\test{Know what bits and bytes are. Know the basic variable types and be
able to give their numerical ranges.}

File/data sizes:

Question: APOGEE image cubes are 2048x2048x47 reads: how much memory?

Question: MULITDARK simulations have 2048x2048x2048 particles with
full phase space information: how much memory?

Physical memory and swap space: code with memory in mind

\test{Be able to calculate how much memory will be used for a
specified amount of data.}

Integer representation: unsigned and signed integers (e.g., two's
complement)

Floating point representation: IEEE, note precision issues

binary vs character representation

architectures and byte-swapping (big-endian vs little-endian)

\subsection{Computing software}
\subsubsection{Operating systems}
Operating system: controls basic interface between user and hardware
\begin{itemize}
    \item Windows
    \item Unix (various ``flavors'', originally BSD vs AT\&T)
    \item Mac OSX (flavor of Unix)
    \item Linux (flavor of Unix)
    \item Others: VMS, ChromeOS, Android, ISO
\end{itemize}
OS must evolve as new hardware is developed, leading to versions, with
implications for stability/support/security. The main core of the
operating system is called the {\bf kernel}, which provides a programming
interface for programs.

Linux implementations (distros): include basic Linux kernel plus
add-on packages, usually with a {\bf package manager}, e.g.\ yum, apt-get.

RedHat / Fedora / CentOS, Debian / Ubuntu, SUSE, Gentoo, others

versions accommodate both OS modifications and package development
\subsubsection{Operating system interfaces}
Operating system generally provides kernel and a core set of commands.
On top of this, there may be command interpreters and/or graphical
interfaces to commands.

Command-line interfaces:

UNIX uses ``shells'' that, in addition to kernel command interface,
allow scripting operations, variables, input/output redirection, etc.
Common shells: csh/tcsh, bash, ksh, each allow for different ``style''

Graphical interfaces: ``window'' system:

Operating system provides hardware interface; most Unix machines use
protocol called X11, Mac OS X uses quartz (but X11 is available)
On top of this, distributions provide window manager / desktop
enviroment, which is a graphical use interface (GUI) to the operating
system
Common desktop enviroments: KDE, GNOME

\test{Understand and be able to articulate the functions/differences
of operating systems, shells, and windowing systems.}

\subsubsection{Applications}
A large amount of software has been developed to work under a given
operating system, and these are available as packages: examples are
editors, compilers, etc. (do a \texttt{rpm -q -a} on one of the Linux machine
to see what is installed). In fact, the shells are installable
applications, so the distinction between core software and
applications is not totally clear.

\subsection{Network communication}
Various software allows for various actions across the network:
\begin{itemize}
    \item ssh: allows for secure login (encrypted communication)
    \item NFS (network file system): allows network disk access
        (at network speeds)
    \item NIS (network information system): allows for sharing of common
        information (e.g., login information, automount services, mail
        aliases, etc.)
    \item CUPS (common Unix printing system) allows for sharing of printers on
        the network
    \item any software can communicate over the network using the general Unix
        concept of sockets, which allow for intermachine communication
\end{itemize}
Remember, network communication is slower than disk communication, and
affects other who are using the network.

\subsection{NMSU/Astronomy (academic)}
Cluster of computers running Linux (CentOS) on Astronomy building
subnet
\begin{itemize}
    \item Some individual user machines, some department machines
    \item CentOS adopted because of security/stability/support issues
    \item Some faculty/students have Apple machines (Mac, Macbook) which run
        MacOSX, another Unix-like variety
    \item Some faculty/students have laptops with some flavor of Linux
        installed (usually Ubuntu?)
\end{itemize}
Servers (most located in room 116A, apart from backup devices in
computer center)
\begin{itemize}
    \item astronomy: NIS master, web server, email server, printer server
    \item astrodisk: disk server (/astro: users, httpd, local, ftp, also
        catalogs, aips, redhat)
    \item astrobackup, astrobackup2: disk backup machines
\end{itemize}
Compute nodes
\begin{itemize}
    \item public machines: hyades, praesepe, virgo
    \item ``private" machines: seismo, solarstorm, milkyway
    \item typically have 16-64 processors, additional memory
\end{itemize}
Desktop machines
\begin{itemize}
    \item (usually) single processors, 4-8 Gby memory
    \item Student machines: 1 TB disk, split into two user partitions:
        /\{machine\} and /\{machine\}-data, with backup implication : files on
        /\{machine\} are backed up (daily for small files, with some history
        retained, weekly for all files, but with no history retained), files
        on /\{machine\}-data are not backed up!
\end{itemize}
Disk sharing is accomplished through NFS\@. The NIS master has a table
of all mounted partitions, and makes them accessible via
/home/partitionname. However, the machine serving the partition needs
to give permission. By default, server partitions are shared to all
machines, and main partition on client machines are shared to the
servers, but client partitions are not shared with other clients
except by request. Local partitions are also available via the NFS
interface, with no penalty, so it is a good idea to use the
/home/partitionname for generic disk access.

Note that the default home directories for students (/home/users/name)
is located on the astronomy server. As a result, if you work in or
under your home directory when logged into your machine, you are using
the network whenever you read/write from disk. It is generally better
to use the local disk if you are doing any significant I/O (note
distinction between /partitionname/ and /partitionname-data/)

Server shares software that is not installable as packages, so using
such software invokes network traffic. Since the server disk is
generally checked whenever you type a command, you may see slow
response at your computer if the network is being excessively taxed.

\test{Understand enough about the NMSU Astronomy setup to recognize
what pices of hardware (CPU, disk, memory, network) you are using
depending on what machine you are logged into, and where you are
reading/writing from/to on disk.}

\subsection{Linux resource usage}
Determining CPU, disk, and memory capability: /proc/cpuinfo,
/proc/meminfo, df, various grapical interface tools (Mac OS X:
system\_profiler)

(note man command to get information on any command!)

machine types, etc.: uname -a

Determining CPU, disk, and memory usage: top, ps, w, du

% 3
\section{Working in a Unix environment}
See tabular summary at \url{unix.html}

\subsection{Linux help}
\begin{itemize}
    \item \url{http://freeengineer.org/learnUNIXin10minutes.html}
    \item \url{http://www.tutorialspoint.com/unix/unix-useful-commands.htm}
\end{itemize}

\subsection{Basics: directories}
File systems: disks (physical units) and partitions (logical units).

\begin{description}
    \item [df] Display free disk space
    \item [rm -r] Recursively remove directories
    \item [rm -i] ``Are you sure?'' Might alias \verb|rm| to this.
    \item [cd -] Go back to directory you were in before the previous
        \verb|cd| command.
\end{description}

Separating file name from directory name: basename and dirname.

\begin{verbatim}
/dev/sda3
/dev/sda5
/acrux - root partition
\end{verbatim}

\subsection{Basics: files}
\subsubsection{Displaying files}
\verb|ls|
\begin{description}
    \item [-a] Include directory names that begin with `.'
    \item [-l] Permissions, \# hardlinks, owner, group, size (bytes),
        date, filename
    \item [-F] Display `/' after directory names
    \item [-r] Reverse order
    \item [-t] Sort by modification time
    \item [-R] Recursive \verb|ls| (all sub-directories)
\end{description}

\subsubsection{Creating files}
\begin{description}
    \item [touch]
    \item [echo] output redirection
    \item [mv]
    \item [cp]
\end{description}

\subsubsection{permissions}
\begin{description}
    \item [chmod]
    \item [chown]
    \item [umask]
\end{description}
Unix permissions cover three types of users:
\begin{itemize}
    \item user, or owner (u)
    \item group (g)
    \item other (o)
\end{itemize}
Each of these can have read,
write, execute permission, expressed in three bits, in order rwx. You
can change permissions using characters, e.g.:
\begin{description}
    \item [chmod o +w] Give \verb|write| permissions to \verb|other|
    \item [chmod g -x] Remove \verb|execute| permissions from \verb|group|
\end{description}
or via a permissions mask.
\begin{verbatim}
755 (rwxr-xr-x)
644 (rw-r-r-)
000 no permission for anyone
101 group and other can execute and read (1) but not write (0)
755 - 111, 101, 101  permission for user to do everything, group and
                     other can't write.
> umask 022 Can put this in .cshrc, second `2' - 010...?
            All permission to owner, everything except write for group
            and other...
\end{verbatim}
If a file is an executable (program), you can run it simply by typing
its name; either need absolute path, or just filename if file is in
the current PATH environment (see below).

\subsubsection{advanced file types}
\paragraph{Links} come in two types:
\begin{itemize}
    \item {\bf symbolic (soft):}
        A symbolic link points to another file, and
        can work across file systems: the link is distinct from the
        destination file, and if the destination file is removed the link will
        be broken.
    \item {\bf hard:}
        A hard link works only within a file system, and provides
        an independent link to the same file: if either source or destination
        is deleted, the file still remains with the linked name.
\end{itemize}
\begin{description}[labelwidth=17em]
    \item [> ln -s \{source\} \{destination\}] Create symbolic link
    \item [> ln \{source\} \{destination\}] Create hard link
\end{description}

\paragraph{Pipes} allow for interprocess communication through a pseudo-file.
They are created using: \verb|mkfifo|

\textcolor{myGreen}{inside a single machine;
program reads from file on disk;
make interface to program on webpage…?;
submit → write input to file.;
Inside program: open file to read from: pipe temporary… .pipefile;
program and website can send information to each other.;
}

\subsubsection{file types}
\begin{itemize}
    \item executable
    \item directories
    \item regular files
    \item links
\end{itemize}

standard file extensions and file naming conventions, e.g.
\url{https://www.openoffice.org/dev_docs/source/file_extensions.html};
\url{https://kb.iu.edu/d/affo}

Directory names: \verb|my\ file|\ldots
Don't use spaces! ``Royal pain in the butt!''

stdin, stdout and stderr: input and output redirection (note shell
dependence)

piping commands: you can direct the output of one command into the
input of another using the | character, e.g. ls | more
pattern matching: wildcards (globbing):
\begin{description}
    \item [*] All strings
    \item [?] Single character
    \item [[abc]] Matches any one, not necessarily in that order
    \item [regular expressions] ?
\end{description}
Re-naming in bulk! Append \verb|.old| to extension, or change the extension:
\begin{verbatim}
foreach x (*.txt)
    mv $x $x.old
    mv $x $x:r.dat
\textcolor{red}{Is this a script? Or typed at command line?}

\subsubsection{locating files}
\paragraph{\verb|find|} ``descends the directory tree'' and beginning at each
pathname (default is current directory) and locates files that meet the
specified conditions. Searches all subdirectories by default!
But nothing upward from
where you are. There’s an option for that? Or just specify pathname as ‘/*’
\end{verbatim}
\begin{itemize}
    \item find ( -name, -type, -mtime, etc.). find will find all files
        under current directory
    \item locate: uses pre-built system database of files
\end{itemize}

file contents:
\begin{itemize}
    \item cat
    \item more
    \item less
    \item od
    \item head
    \item tail
\end{itemize}
You can use `` to put output of a command into another command.
\textcolor{red}{Must be a typo here.}

\test{Be comfortable\ldots}

\subsection{Unix useful file manipulation tools}
\begin{itemize}
    \item grep: search for
        specified pattern within files, useful flags, -i (ignore case), -v
        (search for all lines that do not have the specificed
        pattern).
    \item sed:
        \begin{itemize}
            \item line-by-line processing using regular expressions
            \item e.g., remove lines
                that match specified pattern: sed '/{pattern}/d' {filename}
            \item e.g., do
                search and replace:
                sed 's/{old}/{new}/' {filename}
        \end{itemize}
    \item tr: line-by-line
        processing with simple replacement. Mostly useful to translating
        special characters, e.g., Windows files ending with <CR> <LF> to Unix
        files
    \item sort:
        \begin{itemize}
            \item sorting alphabetically and numerically by column
            \item use -key=column, e.g. sort -key=2 {filename} will sort alphabetically by
                column 2
            \item use -n for numerical sort
        \end{itemize}
    \item paste/join: pasting files together
    \item awk:
        \begin{itemize}
            \item provides a command-line programming interface, convenient for
                simple operations (but can handle more complex ones)
            \item e.g., arithmetic
                based on columns:
                \verb|awk '$1>x {print $2, $3, 5*$4, $6/$3}' {filename}|
            \item can put awk commands in a separate file and use this to execute: awk
                -f {awkfile} {filename}
            \item can use awk on variables, e.g.
                \begin{verbatim}
                cl> echo $var | awk '{printf("set %s = %s.new\n)",$1,$1}'
                \end{verbatim}
        \end{itemize}
    \item diff: show differences between files
    \item wc : counts lines, words, characters
\end{itemize}

\test{Be familiar with \ldots}

\subsection{Archiving and compression tools}

tar :

bundles multiple files into a single archive
%e.g., put files into archive: tar -cvf {archivename} {file}|{directory}
%to extract: tar -xvf {archivename} to extract
gzip : compresses files losslessly, creates .gz files

compress : another compression, creates .Z files

xzip: yet another compression, creates .xz files

fpack: tool for compressing FITS files

\subsection{Unix system information, resources and usage}
\begin{verbatim}
Determining CPU, disk, and memory capability: /proc/cpuinfo,
/proc/meminfo, df, various grapical interface tools (Mac OS X:
system_profiler)
\end{verbatim}

(note man command to get information on any command!)

machine types, etc.: uname -a

Determining CPU, disk, and memory usage: top, ps, w, du

w, whoami

hostname

\test{Understand\ldots}

\subsection{Unix job control}
ps: show current processes

top: show processes sorted by resource usage, updates in real-time

kill: sends signals to processes multiple signals available

CTRL-Z: sends a stop signal to running process

CTRL-

: sends a kill signal to running process

foreground/background: using signals (note signal trapping in code)

cron jobs: allow for job to run on a regular schedule, through use of
a crontab (crontab -e to edit).

\test{Understand\ldots}

\subsection{Intermachine communication}
ssh: used for secure remote login, but also can be used to execute
commands on remote host (e.g., ssh hyades w)

scp: copy files over ssh connected

sftp: open a session with a remote host to enable transfer of one or
multiple files/directories

ftp: older, less secure, method for file transfer, but still often
used for anonymous ftp, where external users can access a restricted
area to grab files, or even to transfer in, if that is enabled. On our
cluster, we run an FTP server on astronomy.nmsu.edu: the reserved area
is under /home/ftp/pub for outgoing file, /home/ftp/incoming for
incoming files.

rsync: used to "sync" files/directories, i.e., transfer only files
that differ between systems. Can run locally or between machines using
ssh protocol (-e ssh)

Globus?

ssh-keys: provides an alternative to password authentication. Instead
of sending a password to a remote machine, a key pair is generated,
with a public and private key. The public key is initially transferred
to the desired remote server; on subseqeuent connection requests, the
server returns the public key to get a match with private key, and if
so, the connection is established. Usually, the matching of the keys
requires a passphrase as well. One advantage is better security.
Another is convenience, as it is possible to register your private key
with an "ssh-agent" so that the passphrase is entered only once for a
session, allowing remote ssh logins for the rest of the session
without needing a password. Commands:
\begin{verbatim}
ssh-keygen
ssh-agent
ssh-add.

Keys are created in /.ssh/; public keys are appended to /.ssh/authorized_keys
\end{verbatim}

\test{Understand how\ldots}

\subsection{Unix environment}

Unix allows for "environment" variables that are visible to all shells
(as opposed to shell variables that are local to a given shell). These
are often used for general configuration, and are very useful in the
context of software setup and package/data management. This is
especially true when you may be setting up an environment on multiple
machines, where root directory names differ.

setenv, printenv: commands to set and show environment variables in
csh/tcsh.

In bash, set environment variables using: export var=value

some common/useful environment variables:
\begin{verbatim}
EDITOR/VISUAL
DISPLAY
SSH_ASKPASS
SHELL
PATH
\end{verbatim}

\test{Understand\ldots}

\subsection{Editors}
An editor is a basic tool used for nearly all computing tasks, so it
is worthwhile to have a strong command of the editor that you choose
to use.

Given today's networked environment, it is very likely that you will
need at some point to edit files on remote machines. Working with
editors that open graphical windows can become a significant challenge
when working over the network, and for this reason, I would strongly
discourage them.

The historical editor associated with Unix is vi/vim, and this is
still in widespread use. Another extremely widespread editor is emacs,
which can be used within a terminal window using the -nw option.
Another terminal based window that is sometimes installed is nano.

With whatever editor you use, you should be able to

efficiently, move the cursor within files. Goto specific line numbers
Search for text strings
Search for text string and replace with alternate string
delete line or blocks of lines with single command
be able to efficiently move blocks of text around, e.g. copy and paste

\test{Be extrememly comfortable\ldots}



\subsection{Unix shells and shell scripting}

startup: .cshrc, .bashrc, with useful customization commands:

alias definitions
environment varibles
set path for searching for executables using PATH environment variable
(printenv PATH to see what path is)
which command: find full directory path of a given command (or find
out whether the command is in the path)

history command: lists past N commands

command completion (TAB-completion) and recall (!N executes command N;
!x executes last command that begins with x.

When to use shell scripts? Often most convenient for operations have
to do with files and simple file modification. Not usually a good
choice for numerical work!

running command files: source vs starting new shell (\#!), file
permissions (a file must have executable permission for it to be
identified in the path as a command).

scripting:

variables: reference using $varname
command line arguments: $0 is first word (the command), $1 the first
argument, etc.; $* refers to all arguments (after command), which is
very useful, e.g., in conjunction with for loops (below) to loop over
all command line arguments
conditionals:
\begin{verbatim}
== (equal)
!= (not equal)
&& (and)
|| (or)
\end{verbatim}
looping

see \url{http://astronomy.nmsu.edu/holtz/a575/unix.html#shell}
for syntax of shell commands in bash and tcsh.

see
\href{http://astronomy.nmsu.edu/holtz/a575/IntroScriptingJuly2015.pdf}
{PDF presentation from Utah CHPC on shell scripts}

csh:
\begin{verbatim}
set
if / then / else
often used with file inquiry conditionals (e.g. -e tests for
existence)
foreach /end
while / end
\end{verbatim}


\subsubsection{Example}

\begin{verbatim}
bsyn &
set bsynjob = $!
set tmax = 300
set runtime = `ps -q $bsynjob -o cputime | tail -1 | awk -F: '{print ($1*3600)+($2*60)+$3}'`
while ( $runtime < $tmax )
  sleep 2
  set runtime = `ps -q $bsynjob -o cputime | tail -1 | awk -F:
  '{print ($1*3600)+($2*60)+$3}'`
  if ( `ps -p $bsynjob -o comm=` == "" ) then
    echo process done, exiting!
    exit
  endif
end
echo expired, killing job
kill $bsynjob
\end{verbatim}

\test{Understand\ldots}

\subsection{Remote \& virtual desktops}
e.g., VNC: vncserver and vncviewer.

% 4. Presentation/communication
\section{Presentation/communication}
\subsection{Text processing: LaTeX}

Basic LaTeX:

\verb|\documentclass{class}| article, letter, book, others (see AASTeX
below, which provides aastex and emulateapj)

Preamble:
\begin{itemize}
    \item package setup: \verb|\usepackage{package}|
        graphicx, amsmath, hyperref, color
    \item command aliases, e.g., \verb|\newcommand{\mh}[0]{[M/H]}|
\end{itemize}
math mode:
\href{https://www.sharelatex.com/learn/Mathematical_expressions}{reference}
\verb|$| and \verb|$$|, superscript, subscript, greek
letters, math symbols, etc. (can be used in iPython notebooks)

various environments: itemize, enumerate, figure, table, tabular,
center, equation, e.g.:
Figures with \verb|\usepackage{graphicx}|, \verb|/includegraphics|
\begin{verbatim}
\begin{figure}
    \includegraphics[width=0.5 \textwidth]{cfig.pdf}
    \caption{caption here}
    \label{fig:cfig} (see below)
    \end{figure}
\end{verbatim}
Refer to Figure [*] in the text to automatically get the correct
number!
\begin{verbatim}
\begin{table}
    \begin{tabular}{llcr}
        obj1 & data1 & text1 & number1 \\
        obj2 & data2 & text2 & number2 \\
    \end{tabular}
    \label{tab:data}
\end{table}
\end{verbatim}
Refer to Table [*] in the text to automatically get the
correct number.

\href{http://journals.aas.org/authors/aastex.html}{AASTeX}:
documentclass for astronomical publications: aastex, emulateapj,
deluxetable environment for tables (including multi-page).

labels: to allow for arbitrary moving/addition/deletion of figures,
tables, sections, etc., do not build in numbering into the text.
Instead use the \verb|\label{labelname}| command to label each entity, and
automatically reference the correct numerical identification using
\verb|\ref{labelname}|. Note that using these will require two passes through
latex, to register the labels correctly. Note that, in the figure
environment, the label command must come after the caption command.

References: use bibitems. Create the reference once, give it an
identifier, and reference it in the text by the identifier using
\verb|\citep, \citet, \citealt| (the natbib package provides an extension
beyond the standard LaTeX commands). You can create the bibitem
manually (\verb|\bibitem{identifier} reference|), or, better yet, use BiBTeX
reference information (available from ADS) and automatically create
the bibitems.
\begin{itemize}
    \item manual bibitem:
        \begin{verbatim}
        \begin{thebibliography}
            \bibitem
            ...
        \end{thebibliography}
        \end{verbatim}
    \item bibtex: create {\tt reffile.bib} file with bibtex entries, use
        \verb|\bibliography{reffile}| in latex file, then
        \begin{verbatim}
        latex file
        bibtex file
        latex file
        latex file
        \end{verbatim}
        to put it all together. See
        \href{http://ads.harvard.edu/pubs/bibtex/}{discussion on ADS};
        for astronomical journal reference style, you may want
        to use the apj bibliography style, using the
        \href{http://ads.harvard.edu/pubs/bibtex/astronat/apj/apj.bst}
        {apj.bst} file.
\end{itemize}

Refer to an article inline (with year in parenthesis) using
\verb|\citet{id}|.
Refer to an article where author and year are both in parenthees using
\verb|\citep{id}|.

\verb|\bibliography{ref}|  at end of document.

``compiling'' LateX / pdflatex. Can use a makefile to simplify. Using
bibtex, the sequence is:
\begin{enumerate}
    \item prepare a {\tt doc.tex} file with the document and a
        {\tt ref.bib} file with a bunch of BibTeX entries
        (e.g., from ADS)
    \item from command line, run
        \begin{verbatim}
        pdflatex doc
        bibtex doc
        pdflatex doc
        pdflatex doc
        \end{verbatim}
\end{enumerate}
latex editors: kile. On line resources for sharing/editing/viewing
LaTeX: sharelatex, authorea, overleaf, etc.

Spell checking: aspell, hunspell. ALWAYS SPELL CHECK YOUR DOCUMENTS!

\test{Be able to easily create a complete LaTeX file that includes
sectioning, tables, figures, and a bibliogrphy, using labels for cross
referencing figures, tables, and sections, and the family of cite
commands for references. Understand how to turn the LaTeX file into a
PDF file, including the sequence needed to include references from a
bibtex file, and section cross-referencing.}

\subsubsection{Introduction to makefiles}
Introduction to makefiles (for example, see the
\href{https://www.gnu.org/software/make/manual/html_node/index.html#Top}
{GNU documentations} or
\href{http://www.rsmas.miami.edu/personal/miskandarani/Courses/MSC321/make.pdf}
{someone else's explanations}).

Standard rules. Standard targets: install, objs, etc.

\subsection{HTML}

\url{http://www.w3schools.com/tags/}\\
\url{http://www.w3schools.com/css/}\\
\url{http://www.ironspider.ca/index.htm}\\
\url{http://www.advancedhtml.co.uk}\\
%\url{}

Structure of an HTML document:
\begin{verbatim}
<command></command> structure, <HTML><BODY [bgcolor=]>,
\end{verbatim}

Basic HTML text processing: Like LaTeX, text will appear continuously in a browser
regardless of how it is entered in the HTML source file, unless there
are command directives. The width of the text will adapt to the width
of the browser. Various text directives:
\begin{verbatim}
    <Hn> Headings in different font sizes (1 is largest) </Hn>
    <P> Force a line break with vertical paragraph space
    <BR> Force a line break without extra vertical space
    &nbsp; Force small horizonal space
    <UL>/<OL> Unordered list / ordered (numbered) list
        (end with </UL> or </OL>)
            <LI>  individual list element
\end{verbatim}
Font commands
\begin{verbatim}
    <I> italic </I>
    <B> bold </B>
    <FONT [color=]> general font command
\end{verbatim}

HTML links:
\begin{verbatim}
    <A HREF=address> active text </A> : sets up a link
    <A NAME=label> : sets a lable in a HTML file, can be directly skipped
        to using address\#label format
\end{verbatim}

HTML tables:
\begin{verbatim}
    <TABLE [BORDER=]>: start a table
    <TR> : start a new row
    <TD [COLSPAN=]> : start a new column
    </TABLE> : end table
\end{verbatim}

HTML figures:
\begin{verbatim}
    <IMG SRC={file} [WIDTH=] [HEIGHT=]>
\end{verbatim}
Force size using \verb|WIDTH=| and/or \verb|HEIGHT=|, usually only one of these,
as this will automatically preserve aspect ratio.
If at all possible, use file format that most browsers know how to
display inline: GIF, JPG, PNG

comments in HTML files: \verb|<!-- comment -->|

All of these can be combined, e.g., a table of figure thumbnails with
links to full size figures:
\begin{verbatim}
<TABLE BORDER=2>
<TR> <TD> <A HREF=fig11.jpg> <IMG SRC=fig11.jpg WIDTH=200> </A>
<TR> <TD> <A HREF=fig12.jpg> <IMG SRC=fig12.jpg WIDTH=200> </A>
<TR> <TD> <A HREF=fig21.jpg> <IMG SRC=fig21.jpg WIDTH=200> </A>
<TR> <TD> <A HREF=fig22.jpg> <IMG SRC=fig22.jpg WIDTH=200> </A>
</TABLE>
\end{verbatim}

Frames, CSS, and more.

\test{Understand how to write a basic HTML page, including lists,
figures, tables, and hyperlinks.}

A web server is a network program that runs on a server machine, and
interprets/delivers web pages as requested. Standard web requests are
http://\{machinename/address\}/\{file|directory\}. One the web server, the
\{file|directory\} is interpreted as a relative path to some document
root directory. If the requested address is a directory, the web
server will look for a file named index.html in that directory and, if
it exists, will use the contents of that file; if it doesn't exist, it
will show the directory listing of the directory.

Web server at NMSU uses /home/httpd/html as the document root, and all
users have a directory /home/httpd/html/\{username\}, so web requests to
http://astronomy.nmsu.edu/\{username\} will look for files in this
directory. Note that this directory is located on the disk server, so
it is not a great location to put large files; if you want to provide
access to large files, or even whole directories, located on a disk on
your machine, consider the use of symbolic links.


\subsection{Collaboration}

email/listservers : value as email archive/ risk of junk mail. Be
aware of reply-all. Listservers as archives. Listserver options:
digest delivery.

File size and format (e.g., for plots): deliver files so as to make it
as easy as possible for audience to read/view/digest them!

web pages: very useful as hierarchical source of information, with
figures

wiki : centralized set of user-editable web pages, with many other
features\@.

\href{http://astronomy.nmsu.edu/holtz/a575/ipynb.html}
{iPython notebooks}.

% 5. Programming
\newpage
\section{Programming}
\subsection{Overarching concepts}
\begin{itemize}
    \item Computing/software is lifelong learning
    \item do as little as possible ``interactively'' so it is
        repeatable.
    \item less is more (to a point); don't repeat work, make the tool
        better and reusable.
    \item all software should be written as if someone else will use
        it: document and share.
\end{itemize}
\subsection{Philosophy}
From Quora, response to a question about value of formal training in
programming vs.\ picking it up on your own:

Upvoted by Alagunarayanan Narayanan, Software Engineer ``I was a
self-taught programmer for about 4 years, before taking a few courses
as extras in my EE degree. I am now pursuing a MSc in CS\@. So my answer
is mostly observations on how I do things differently now compared to
how I did them before.

Architecting projects is a big thing. Given a list of things the
software is required to do, how would you lay out your program so that
it
\begin{enumerate}
    \item Does what it's supposed to do
    \item Is maintainable
    \item Is easily understandable for other people (and yourself in a few
        months)
    \item Is easily extensible if the requirements change (as they always do)
        Uses the most suitable design patterns to make the code intuitive
\end{enumerate}
Usually self-taught programmers do 1., but don't pay nearly as much
attention, or only pay lip service to the other points. I thought I
knew about all those other points, but as I found out later, I really
had no idea.

Another big difference is how much trial and error. I did a lot more
trial and error back in the days, because I didn't have good
understanding about how ``the whole stack'' works. I used a lot of
things without bothering to truly understand how they work under the
hood.

Nowadays I almost don't do any trial and error at all. I think through
everything, and most things I actually start coding actually work on
the first try (not counting typos, etc). I know exactly how memory
management works all the way from kernel to malloc. I know how
schedulers work (having written a few for a course), and I know what
the synchronization mechanisms are, the pros and cons of each, the
usual patterns in which they are used, as well as how they are
actually implemented on the instruction level. As a self-taught
programmer, I knew how to use mutexes and that's about it. As it
turned out, I actually (poorly) reinvented a few of the other standard
mechanisms.

Self-taught programmers are also usually not used to reading a lot of
code written by other people, which is a very important skill when
working in teams.

Knowledge of the existing algorithms is another one.

When a non-trivial problem is encountered, a bad programmer dives head
first into coding a solution. A better programmer looks for solutions,
and tries them out. A good programmer looks for solutions, analyzes
them for time and space complexity as well as other constraints, and
implements the most likely one.

Most self-taught programmers start coding too early.

And like you said, things like AI and ML\@. Most self-taught programmers
never learn those things, because usually they only learn things they
need, and if you don't know AI and ML, they won't seem like possible
solutions to your problems, and you'll never think about learning
them. It's a chicken and eggs problem. Self-taught programmers often
don't know what they don't know. One nice thing about doing a degree
is that it almost forcefully introduces you to everything, so by the
time you are done, at least you know what you don't know.''

\subsection{Languages}
Compiled vs non-compiled. Distinction perhaps less clear now than it
was in the past. But key point is that, for certain applications, you
may want to consider execution time.

Open-source vs proprietary

Languages: commonly used in astronomy: Python, C, C++, Fortran, IDL,
MATLAB. Many other languages exist: C++, C\#, Java, Javscript, etc.
etc.

Fortran: historical language of choice of scientific computing in
mid/late 20th century. Many codes developed that are still in
existence/development (e.g., Anatoly's N-body and Hydro codes,
synthetic spectra generation MOOG and TURBOSPEC, stellar evolution
MESA, Chris' codes, others...). In addition, many routines coded for
fast operation available, e.g. LAPACK. Various versions of fortran:
F66, F77, F90/F95

C: foundational language for computer science. Some astronomical
routines: SAO WCSTOOLS for astronomical coordinate system routines,
HSTPHOT/DOLPHOT for HST photometry, SLALIB for more astronomical
coordinate routines, others....

Python: major push in current astronomical software development.
astropy

IDL: historical very close connection to astronomy, with much
development at LASP and Goddard Space Flight Center in the public
domain. Marketed through RSI, then ITT, now Exelis, as a licensed
software product. Originally written in C? Major astronomical software
infrastructure exists in IDL, e.g. for planetary and solar
astrophysics (perhaps because of NASA / GSFC). Astronomy users
library. Major component of SDSS software, although modern evolution
away from it.

\subsubsection{Getting started}
see \href{http://astronomy.nmsu.edu/holtz/a575/programming.html}
{programming reference table}.

\subsubsection{Getting started with Fortran}
Basic program structure:
\begin{enumerate}
    \item program statement
    \item variable declarations
    \item program statements
    \item end
\end{enumerate}
Basic language:
\begin{itemize}
    \item f77 starts in column 7, f95 can start in column 1
    \item Comments: `!' (f95) `C' in column 1 (f77)
    \item line continuation: `\&' (f95 and f77)
\end{itemize}
Compiling and running programs: compiling, linking, makefile, BINDIR,
path

\subsubsection{Getting started with C}
Basic program structure:
\begin{enumerate}
    \item \#include files
    \item int main()\{
    \item variable declarations
    \item program statements
    \item \}
\end{enumerate}
Comments: `//' or embed between `/*' and `*/'.

Compiling and running programs: compiling, linking, makefile, BINDIR, path
\subsubsection{Introduction to makefiles}
\url{https://www.gnu.org/software/make/manual/html_node/index.html#Top},
\url{http://www.rsmas.miami.edu/personal/miskandarani/Courses/MSC321/make.pdf}

Explicit commands, variable filenames, rules.

Defining rules, e.g.\ Latex into PDF, etc Standard rules.

Standard targets: install, objs, etc.
\subsubsection{Getting started with Python}
\paragraph{Using command interpreter:}
\begin{itemize}
    \item python
    \item ipython \href{http://ipython.org/ipython-doc/2/interactive/}
        {features}
\end{itemize}

\paragraph{Using programs:}
running interactively, running as a script, running as
an executable, using \%run with ipython

\paragraph{Environment variables:}
\begin{description}
    \item [executables] searched for in PATH
    \item [imports] searched for in PYTHONPATH,
        as well as system-installed libraries.
\end{description}
To edit PYTHONPATH:
\begin{verbatim}
export PYTHONPATH=${PYTHONPATH}:/path/to/directory/
\end{verbatim}
\begin{verbatim}
>>> import sys
>>> sys.path.append(``path/to/Modules'')
>>> print sys.path
\end{verbatim}
In order to have Python see the modules inside each subdirectory,
add a blank file called \verb|__init__.py| to each subdirectory
(with two underscores on each side).

To automatically import the usual, edit the file:
\begin{verbatim}
cl> vi ~/.ipython/profile_default/startup/00startup.ipy
  import numpy as np
  import matplotlib.pyplot as plt
  ... etc.
\end{verbatim}
This will automatically be executed when \verb|ipython| is started.
The numbers in front of the file name specify the order of preference
in which they are executed, e.g.\
\begin{verbatim}
00first.ipy
50middle.ipy
99last.ipy
\end{verbatim}

Comments: hash (\#).

Python useful references: \href{http://www.python-course.eu/course.php}
{Python tutorial},
\href{https://python4astronomers.github.io}{python4astronomers}


\subsubsection{Getting started with IDL}
Using command interpreter.
\paragraph{Using programs:}
run a script using \verb|IDL> @scriptname|, run a program using
\verb|IDL> .run programname| or
\verb|cl> idl -e|

IDL\_PATH

Comments: semicolon (;)
\subsubsection{Simple makefile}
Simple makefile (note that indents must be TABs):

\begin{verbatim}
hello_f:
    f95 -c hello_f.f95
    f95 -o hello_f hello_o

hello_c:
    cc  -c hello_c.c
    cc  -o hello_c hello_c.o

run:
    hello_f
    hello_c
    hello.py
    idl -e ``.run hello.py''
\end{verbatim}

\test{Understand how to write basic programs in Fortran, C,
Python, and IDL. KNow how to write and use a basic Makefile for
compiling and linking Fortran and C programs.}

\subsection{Basic Programming}
See \href{http://astronomy.nmsu.edu/holtz/a575/programming.html}
{programming reference table} for syntactical details in each
language.
\subsubsection{Program construction}
basic program structure,
comments and good commenting practice,
line continuation
\subsubsection{Variables}
variables and memory, declaration and initialization.
Basic variable types: integer, float, string. Determining variable
type in Python (type) and IDL (size,/type).
Type conversion
arrays and array operation: order of elements in multidimensional
array (column-major vs row-major). Array as a continuous stretch in
memory.
pointers: addresses vs values
dynamic memory allocation: allocate and malloc
multitype collections: structures (derived data types). Arrays of
structures. Python lists, tuples, dictionaries, and structured arrays

\test{Understand the basic variable types and difference between
languages that require variables to be declared and those in which the
type is determined dynamically}.

\subsubsection{Operators}
Mathematical, string, bitmasks and logical operators, vector operators
\test{Understand basic operators and how to use them. Understand
bitmasks and logical operators. Understand how vector operators can be
used in some languages.}
\subsubsection{Control statements}
Conditional: if/then/else, control loops: for and while.

\test{Understand and be very familiar with how to use control
statements in a programming language (Python recommended),
and how to at least recognize and understand the funtionality
of such statements in other languages.}
\subsubsection{Input and output (I/O)}

* I/O : formatted output.
* I/O : simple input. Reading unknown length files in Fortran and C
* I/O : higher level routines: Python astropy.io.ascii, numpy.loadtxt
(but note this only reads a single data type into an array, rather
than into a structured array, as astropy.io.ascii does). IDL
read\_ascii
* Binary data: note issues of N-dimensional array unwrapping, byte order
Fortran: OPEN(lun,file,FORM='binary'), READ/WRITE without format
statements
C: fopen(), fread(), fwrite()
Python: open files in binary mode (rb, wb), struct.pack and
struct.unpack to convert to binary data
IDL: READ\_BINARY
* FITS files
image data: headers and data. N-dimensional images. Standard header
cards. WCS.
FITS tables
Extensions/HDUs
Routines: Fortran/C
\href{http://heasarc.gsfc.nasa.gov/fitsio/fitsio.html}
{cfitsio},
\href{http://astropy.readthedocs.io/en/latest/io/fits/index.html}
{Python astropy.io}, IDL
\href{http://idlastro.gsfc.nasa.gov/contents.html#C9}
{Users library FITS}
routines, e.g., mrdfits/mwrfits
* note Python astropy.table unified I/O for multiple file types,
including ASCII and FITS!
* Databases as alternative to data files

\href{https://python4astronomers.github.io/files/files.html}
{Python4astronomers primer on reading and writing files}

\test{Understand and be very familiar with how to read and write files
in a programming language (python recommended) and how to at least
recognize and understand the functionality of such functions in other
languages.}
\subsubsection{Program organization: subroutines and functions}
Utility of functions:
    improve readability of code,
    minimize/eliminate code repetition: more compact code and easier to
    change,
    use same functions in multiple programs

Using libraries in Fortran, C, Python, IDL:

In many cases, you may have functions that you wish to use with
multiple main programs. In this case, rather than including the same
code in multiple program files, you want to keep the code in separate
``utility'' files, where these files contain only
functions/subroutines, but not main programs.

Python: modules. from and import statements. Location of files: system
installation and PYTHONPATH environment variable.
\begin{verbatim}
import name
import name.subdir
import name.subdir as subdir
from name import subdir
from name import *
\end{verbatim}
Suggestion: collect all of your Python routines under a python
directory and include this in your PYTHONPATH, so you can find all of
your Python routines in one place (no need for ``I know I did this
somewhere, but can't find where\ldots'')

Examples of some libraries:
\begin{itemize}
    \item Python:
        \href{https://docs.python.org/2/library/index.html}
        {standard libraries}, including
        \href{https://docs.python.org/2/library/os.html}
        {os (operating system routines)}, add-on libraries, e.g.,
        \href{http://www.astropy.org}
        {astropy},
        \href{http://www.numpy.org}
        {numpy}
        \href{http://www.scipy.org}
        {scipy} (wide range of scientific/numerical techniques), and
        many others.
\end{itemize}

\test{Understand \ldots}

\test{Understand \ldots}

\subsubsection{Objects and object-oriented programming}

traditional programs tend to think of variables and functions that act
on variables. Object oriented program tends to think of objects that
can have associated attributes, but also actions that may depend on
the attributes.

Compare dictionary (or structured array) with a class
Objects: attributes and methods.
Python:
everything is an object! E.g.,
\href{https://docs.python.org/2/library/stdtypes.html#string-methods}
{strings},
\href{https://docs.python.org/2/tutorial/inputoutput.html#methods-of-file-objects}
{files} as objects.
Inspecting attributes and methods in Python using iPython:
object.<tab>, object.function?. Regular Python: dir() function.
Defining objects with CLASS, e.g.,
\begin{verbatim}
h0 = 72.

class gal :
    def __init__(self) :
        self.name = 'test'
        self.ra = 120.
        self.dec = 40.
        self.vrad = 3000.

    def dist(self):
        return self.vrad / h0
\end{verbatim}
Note that object inheritance is possible, example:
isochrone as a collection of stars

\test{Understand \ldots}

\subsection{Error handling and code testing}

error handling: good code will trap errors and report the source
rather than rely on the error message when the code crashes. Note
Python try/except statements.

testing code: find a case where you know the solution and make sure
your code gets the right answer! Think of challenging cases that you
can test, or at least test for behavior in the expected direction.

\subsection{Debugging code}
Python: \href{https://docs.python.org/2/library/pdb.html}{pdb}

\subsection{Coding practices}
style, e.g. PEP8 guidelines, e.g., white space, indentation rules, ...
Make your program readable. Consider white space judiciously,
recognizing tradeoff between separating program blocks and increasing
length of code. Modular code is generally easier to read and digest.
comments: every program must be commmented! Include an overview
comment at the top and clarification comments throughout the code as
needed. Don't need to comment what is apparent from reading the code
if the code is straightforward. Separate comment statements are
usually preferable to ``end-of-line" comments.
documentation and documentation tools: Sphinx, doxygen (examples)
Avoid hard-wired directories: consider use of environment variables to
specify root directories
Avoid hard-wired constants, file lengths
version control and package management: tagging in version control
software; modules: handles environment variable setup, dependencies,
etc. Example\ldots

\test{Understand \ldots}

\subsection{More advanced coding}
Command line arguments
Signal trapping
sockets
Parallel programming
cross-language programming

\subsection{Case study/review}
Consider isochrone reading code examples (isochrones.py)
Example of astropy FITS I/O.

% 6. Plotting
\newpage
\section{Plotting}
Interactive plotting vs.\ subroutine plotting.

\href{http://www.star.bris.ac.uk/~mbt/topcat/}{TOPCAT}

Python matplotlib and IDL

\subsection{Basic Plots}
Line graphs: colors, line styles, limits, labels, multiple data sets
on a single plot

Point graphs: colors, point types, limits, labels

Hardcopy


\subsubsection{Python}
See \href{http://matplotlib.org/faq/usage_faq.html}{matplotlib usage}
Parts of a plot.

\href{http://matplotlib.org/users/pyplot_tutorial.html}
{tutorial}

\href{http://matplotlib.org/api/pyplot_summary.html}
{plot commands}

State machine interface:

Line plots:

\begin{verbatim}
import matplotlib.pyplot as plt #(already done with ipython --matplotlib)
plt.plot(y)
plt.plot(x,y)
plt.plot(x,y,color='red'|'green'|'blue'|'cyan'|'magenta'|'yellow'|'black'|....)
plt.plot(x,y,linestyle='-'|'--'|'-.'|':',linewidth=1)
plt.plot(x,y,drawstyle='steps-mid')  # "histogram" style connect
plt.plot?  # for more options
\end{verbatim}

Plot points:
\begin{verbatim}
plt.plot(x,y,marker='o'|'.'|...)
plt.plot(x,y,'r-'|'go'|'b.'|....)
\end{verbatim}

For all sorts of point/marker types, see
\href{http://matplotlib.org/api/markers_api.html}
{markers}.

Error bars:
\begin{verbatim}
plt.errorbar(x,y,xerr=xerr,yerr=yerr)
plt.errorbar?  # for more options
\end{verbatim}
where xerr and yerr can be single values to be used for all points, or
arrays of values for separate error bar lengths for each point.

Limits and labels:
\begin{verbatim}
plt.xlim([xmin,xmax])
plt.ylim([ymin,ymax])
plt.xlabel(label)
plt.ylabel(label)
plt.text(label[0],label[1],label[2])
\end{verbatim}

Hardcopy (note that if you want to make hardcopy plots without having
a plotting window opened, don't use ipython, just use python.):
\begin{verbatim}
plt.savefig(file)
\end{verbatim}
Alternatively, there is a pyplot object interface, which is claimed to
offer more flexibility (e.g., can have multiple plots open):
\begin{verbatim}
fig = plt.figure()
ax = fig.add_subplot(1,1,1)
ax.plot(x, y)
ax.set_xlim([xmin,xmax])  # note variant on state method command
ax.set_ylim([xmin,xmax])  # note variant on state method command
plt.set_xlabel(label)     # note variant on state method command
plt.set_ylabel(label)     # note variant on state method command
plt.draw()                # required in object mode to draw plot
plt.show()                # needed to show figure in non-interactive
mode
plt.savefig(file)
\end{verbatim}
Many data sets in astronomy are ``multidimensional'', and it can be
powerful to display additional information on point graphs using color
and point size:
\begin{verbatim}
scat=ax.scatter(x,y,c=z,vmin=vmin,vmax=vmax,s=size,cmap=cmap,linewidth=linewidth)
\end{verbatim}
where z and size can be arrays of colors (or values to be scaled using
vmin and vmax) and point sizes. If you are mapping a range of z
values, the vmin= and vmax= keywords give the minimum and maximum data
values; all of your data will be scaled into an 8-bit range between
these two values. These will then be displayed in ``color'' depending
on the color map that is specified; a color map determines what colors
are used for each of the scaled data values. The default matplotlib
color map is `jet', which scales things from blue to red, but
\emph{many} different \href{http://matplotlib.org/users/colormaps.html}
{colormaps} are available, and it is possible to define your
own. Note that there is fair amount of literature and opinion on good
choices of color map (see, e.g.\
\href{https://jakevdp.github.io/blog/2014/10/16/how-bad-is-your-colormap/}
{here}), and that most believe that the
default `jet' colormap is a poor choice.

Exercise: plot isochrone color coded by age, with radius coded by
point size

Multipanel plots:
\begin{verbatim}
plt.subplot(2,1,1)
plt.plot(x,y)
plt.subplot(2,1,2)
plt.plot(x2,y2)
plt.subplots_adjust([hspace=x],[wspace=x],[sharex=True])
# need to control axis labels?
plt.xticks([])
#alternatively, using axis objects
fit=plt.figure()
ax=fig.add_subplot(2,1,1)
ax.plot(x,y)
# or to get a grid of axes objects in a single command:
f, ax=plt.subplots(ny,nx)
ax[iy][ix].plot(x,y)
ax[iy][ix].set_xticks([])
ax[iy][ix].set_visible(False)
\end{verbatim}
A useful Python command for ``packing'' the plots onto the page:
\begin{verbatim}
plt.tight_layout()
\end{verbatim}
(also available through the subwindows button on the matplotlib
window).

Exercise: plot temperature vs log g, absolute mag on two plots,
sharing x axis.

For even more flexibility, see
\href{http://matplotlib.org/users/gridspec.html}{gridspec}.

Histograms: binning, limits, labels. Python
\href{http://matplotlib.org/api/pyplot_api.html#matplotlib.pyplot.hist}
{hist}.
\begin{verbatim}
plt.hist(data,[bins=n|bins=binarray])
\end{verbatim}

\test{Understand\ldots}

\subsection{``Image'' plots}

It is often the case that you have 3D information, i.e., a variable as
a function of two other variables. A common example is an image, which
is a array of intensities at a grid of different (x,y) pixel
locations. But there are many other possibilities as well, e.g., the
value of some function across a 2D parameter space.

3D data/images can be displayed in multiple ways. One such way is to
encode the variable as a color or intensity, i.e., visualize the data
as an image. In Python, this is done using the
\href{http://matplotlib.org/api/pyplot_api.html#matplotlib.pyplot.imshow}
{imshow} command. When
you display an image, you are displaying data as a function of each
(x, y) pixel position. It is likely that these pixel position
represent some physical quantity (e.g., position on the sky, value of
some independent variables, etc.), so you may want to associate this
physical quantity with the pixel positions; this can be done at the
plotting level using the extent= keyword in imshow.

Note that imshow by default may attempt to ``smooth'' your image; this
behavior can be avoid by using the interpolation=`none' keyword.
\begin{verbatim}
# generate "unit" grids, note useful numpy.mgrid
y, x = np.mgrid[0:100,0:100]
# alternatively, generate grids with fixed number of points between two values
y, x = np.mgrid[0:5:200j, 3500:5500:200j]
plt.imshow(y,[vmin=],[vmax=],[extent=ymin:ymax,xmin:xmax],[interpolation='none'])
plt.colorbar
\end{verbatim}
The ``unit'' grids can be used to generate functions of two variables,
e.g.
\begin{verbatim}
r2=x**2+y**2
plt.imshow(r2,[vmin=],[vmax=],extent=[ymin:ymax,xmin:xmax])
\end{verbatim}
Exercise: display vmacro relation across HR diagram. Add isochrone
points. Add observed data points.

An alternative way to display 3D images is via a countour plot, e.g.
\href{http://matplotlib.org/examples/pylab_examples/contour_demo.html}
{matplotlibcontour} plots:
\begin{verbatim}
cs=plt.contour(x,y,z)
plt.clabel(cs)
\end{verbatim}

\test{Be able to generate a simple image plot from a 2D array of data
values.}

\subsection{Other plots}
\subsubsection{IDL examples}
\subsection{Event handling}
It can be very useful for a program to be to interact with a plot,
i.e., for the program to take some action based on input from the
plot. This can be accomplished by setting up an ``event handler,''
which provides code that responds to events such as mouse click, mouse
motion, key press, etc.

See \href{http://matplotlib.org/users/event_handling.html}
{matplotlib event handling}

For example, in matplotlib windows, a basic event hander is set up
that displays the position of the cursor in the plot window. But this
can be extended, e.g.:
\begin{verbatim}
fig = plt.figure()
ax = fig.add_subplot(111)
ax.plot(x,y)

#define an event handler to work with key presses
def onpress(event):
    print 'key=%s, x=%d, y=%d, xdata=%f, ydata=%f'%(
        event.key, event.x, event.y, event.xdata, event.ydata)

cid = fig.canvas.mpl_connect('key_press_event', onpress)
\end{verbatim}
This prints to screen, but what if you want to get information back to
a calling routine? Example: say you are fitting a relation to a data
set, and recognize that outliers are throwing the fit off: you want to
identify these based on a residual plot. Set up blocking and return of
data, e.g.:
\begin{verbatim}
def onpress(event):
    ''' the event handler '''
    # define variables as global so that they can be returned
    global button, xdata, ydata
    print 'key=%s, x=%d, y=%d, xdata=%f, ydata=%f'%(
        event.key, event.x, event.y, event.xdata, event.ydata)
    # load the global variables
    button = event.key
    xdata = event.xdata
    ydata = event.ydata
    # once we have an event, stop the blocking
    fig.canvas.stop_event_loop()

def mark() :
    ''' blocking routine to wait for keypress event and return data'''s
    # start the blocking, wait indefinitely (-1) for an event
    fig.canvas.start_event_loop(-1)
    # once we get the event, return the event variables
    return button, xdata, ydata

# call the blocking routine
    button, xdata, ydata = mark()
\end{verbatim}

Generate random sequence (numpy.random.normal) with an outlier, and
identify it from a plot.

% 7. Algorithms
\newpage
\section{Algorithms}
\subsection{Writing a Program}
Simple suggestions:
\begin{enumerate}
    \item Make sure you fully understand the question/problem before
        starting to write the code. Outline the methodology you will
        use in words, diagrams,
        or figures, before getting caught up in the syntax of coding.
    \item Generate the code in pieces. Consider writing out all the comments
        before the actual coding. Include some debugging statements, considering
        the possibility of building in a ``verbosity'' level from the start.
        Consider building in error trapping from the start.
    \item Test code. Use it in an unexpected way to see what happens. Do your
        best to find what is wrong with it. Try to break it.
    \item Clean up code. Consolidate repeated code, find a more efficient
        or transparent way to write things.
    \item Fully comment code. Document the overall strategy and points of the
        code that are not immediately understandable from the code itself.
\end{enumerate}

\subsection{Speed and scaling}
\subsection{Lists and list matching}
(sorting, finding items, matching lists, etc.)

\subsection{Random number generation}
(see NR chapter 7).

A random distribution of numbers is useful for simulating data sets to test
analysis routines. Research your random number generator if what you're doing
is important since crappy ones exist.

Computer generated random numbers are repeatable. They generally start with a
\emph{seed number} (usually the current clock time) if the user doesn't specify
one. Users should record whichever seed they use in case repeatability is
desired.

Lowest level random number generators give \emph{uniform deviates},
random numbers that lie within a specified range.
(usually 0 to 1 for floats, or 0 to 2$^{32}-1$ or 2$^{64}-1$ for integers).

Python stuff:
\begin{verbatim}
>>> random.random
>>> random.seed
>>> numpy.random.random
>>> numby.random.seed
\end{verbatim}

{\bf Transformation Method}:
To generate random numbers for some other distribution, such as a
Gaussian, Poisson, luminosity function, or mass function:
consider a cumulative distribution of the desired function (whose values range
between zero and one). Generate a uniform random deviate between 0 and 1; these
are your y-values. Solve your function for $x$ in terms of $y$, and calculate
all the values of $x$ that correspond to your $y$-values (the random numbers).
This does require that you can integrate your function, then invert
the integral.  (see NR, figure 7.3.1)

In-class exercise: generate random deviates for a ``triangular'' distribution:
$p(x) \propto x$. What does the constant of proportionality need to be in order
to make the integral equal to 1 (aka: a probability distribution)?
\begin{align*}
    \int{p(x) dx} &= \int{Cx dx}\\
    &= C\int{x dx}\\
    C &= 2
\end{align*}
Use the
relation to generate random deviates, and plot them with a histogram.
\begin{align*}
    y &= 2x \\
    F &= \int \! y \ \mathrm{d}x = x^2\\
    x(F) &= \sqrt{(F)} \\
\end{align*}
{\bf Rejection Method:}
If you can't integrate and invert your function:
choose a function that you \emph{can} integrate and invert
that is always higher than your desired distribution.
This is the comparison function, $c(x)$. As before, choose a random deviate
and find the corresponding $x$ values. Calculate the ($x$) value of both
the desired function and comparison function. Choose a uniform deviate between
0 and $c(x)$. If it is larger than $f(x)$, reject your value and start again.
This requires two random deviates for each attempt.
The number of attempts
before you get a deviate within your desired distribution depends on how
close your
comparison function is to your desired function (see NR figure 7.3.2).

Example demonstration: truncated Gaussian distribution. Generate random
deviates with rejection method.

\test{Understand how to use and implement the transformation method for
getting deviates from any function that is integrable and invertible.}

See NR for several ways to generate deviates in desired functions.

{\bf Gaussian distribution:}
Applies to many physical quantities, such as the Maxwellian speed distribution,
and in general as a reasonable approximation for a large mean.

\href{http://docs.scipy.org/doc/numpy/reference/generated/numpy.random.normal.html}
{\tt >>> numpy.random.normal}
\begin{equation*}
    P(x,\mu,\sigma) = \frac{1}{\sigma\sqrt{2\pi}}
    e^{-\frac{1}{2}(\frac{x-\mu}{\sigma})^2   }
\end{equation*}

{\bf Poisson distribution:}
Used for counting statistics.

\href{http://docs.scipy.org/doc/numpy/reference/generated/numpy.random.poisson.html}
{\tt >>> numpy.random.poisson}
\begin{equation*}
    P(x,\mu) = \frac{\mu^xe^{-\mu}}{x!}
\end{equation*}

In class exercise: simulate some data, e.g.\ a linear relation (x=1-10, y=x)
with a Gaussian (mean=0, sigma=varius) scatter, and plot it.
Alternate: CMD from
isochrones with Poisson scatter in colors and magnitudes.

\test{Know how to use canned functions for generating uniform deviates,
Gaussian deviates, and Poisson deviates.}

\subsection{Interpolation}
(see NR chapter 3)

Interpolation can be used if you have a tabulated or measured set of
data, and want to estimate the values at intermediate locations in your data
set, such as inverting a function (e.g.\ random deviates as above,
wavelength calibration, etc.), re-sampling data (images or spectra), etc.
\begin{itemize}
    \item {\bf Linear interpolation:}
        This is the simplest interpolation. To find the
        value $f(x)$ at some intermediate location
        $x$ between two points, $x_{i}$ and $x_{i+1}$:
        \begin{align*}
            f(x) &= Ay_i + By_{i+1} \\
            A &= \frac{x_{i+1}-x}{x_{i+1}-x_i} \\
            B &= \frac{x-x_i}{x_{i+1}-x_i}
        \end{align*}
    \item {\bf Polynomial interpolation:} (NR 3.2)
        The {\it Lagrange formula} gives the value of a polynomial
        of degree $k$ going through $k+1$ points at some arbitrary position,
        $x$, as a function of the tabulated data values:
        \begin{align*}
            L(x) &= \sum ^k_{j=0} \ y_j \ell_j (x) \\
            \ell_j(x) &= \prod _{0 \leq m \leq k, m \ne j}
            \frac{x-x_m}{x_j - x_m}
        \end{align*}
        Note that this provides the values of the interpolating polynomial at
        any location without providing the coefficients of the polynomial itself.
        The coefficients are determined such that the polynomial fit goes exactly
        through the tabulated data points (the measured data).
\end{itemize}
Be aware that higher order is not necessarily better. A classic example:
$$ f(x) = \frac{1}{(1+25x^{2})} $$
Higher order is bad here because\ldots?

There are also issues with even orders because this leads to using more tabulated
points on one side of the desired value than the other
(since even order means odd number of data points).
At high order, the polynomial can go wildly off, especially near data edges.

Another problem with piecewise polynomial interpolation, even at low order,
is that the input data values that are used to interpolate to a desired
location change as you cross each data point. This leads to abrupt changes
in derivatives (infinite second derivative) of the interpolated function.
For some applications, this can cause problems, for example,
if you are trying to
fit an observed data set to some tabulated series of models and are
using derivatives to get the best fit.

{\bf Spline interpolation:}
This issue can be overcome using spline interpolation, which
gives interpolated values that go through the data points but also
provides \emph{continuous second derivatives}. Doing so, however, requires
the use of non-local interpolation, i.e. the interpolating function
includes values from all tabulated data points.  Generally, people use
cubic fits to the second derivatives, leading to \emph{cubic spline
interpolation}. To do this, specify boundary conditions at the ends of
the data range. Usually, a \emph{natural spline} is adopted, with zero
second derivatives at the ends, but it is also possible to specify a
pair of first derivatives at the ends.

Cubic splines are probably the most common form of interpolation,
though not necessarily the best in all circumstances.

Python implementation:
\href{http://docs.scipy.org/doc/scipy/reference/generated/scipy.interpolate.interp1d.html#scipy.interpolate.interp1d}
{\tt scipy.interpolate.interp1d} calculates
interpolating coefficients for linear and spline interpolation, and can then
be called to interpolate to desired position(s):
\begin{verbatim}
>>> from scipy import interpolate
>>> intfunc = interpolate.interp1d(xdata,ydata,
        kind=`linear'|`slinear'|`quadratic'|`cubic'|order)
>>> intfunc(x)  # returns interpolated value(s) at x
\end{verbatim}

Exercise:
\begin{enumerate}
    \item Define some function, $f(x)$\\
        \verb|def f(x):|
            \ldots Your function here\ldots
    \item For some given domain, plot your function at high sampling, e.g.:
        \begin{verbatim}
        # get a high sampled array of independent variable
        x = numpy.linspace(xmin,xmax,1000)
        # plot the relation as a smooth line
        plt.plot(x,f(c))
        \end{verbatim}
    \item For some given domain, sample your function with $N$ points
        (evenly spaced, or maybe random), e.g.:
        \begin{verbatim}
        np.linspace(xmin,xmax,n) # provides n xdata points in your domain
        plt.plot(xdata,func(data),`o') # plot data points as points
        \end{verbatim}
    \item Determine the interpolating polynomial for linear and cubic
        spline, and overplot the interpolated function.
        \begin{verbatim}
        from scipy import interpolate
        # get the interpolating function (you need to choose which type)
        intfunc = interpolate.interp1d(xdata,func(xdata),
            kind = `linear'|`slinear'|`quadratic'|`cubic'|order)
        # plot interpolated data at high sampling
        plt.plot(x,intfunc(x))
        \end{verbatim}
\end{enumerate}

\test{Understand what is meant by piecewise polynomial interpolation and
spline interpolation and under what circumstances it might be better to use
the latter.}

\subsection{Fourier analysis basics and sinc interpolation}
(see NR chapter 12)

Consider a function in two representations: values in
``physical'' space (e.g., time or location) vs.\ values in ``frequency'' space
(e.g., temporal frequency or wavenumber). The two are related by Fourier
transforms:
\begin{align*}
    H(f) = \int \! h(x)e^{-2\pi ifx}\ \textrm{d}x\\
    h(x) = \int\! H(f)e^{-2\pi ifx}\ \textrm{d}f
\end{align*}
Note that different implementations use different sign conventions,
for example: \emph{angular} frequency ($\omega = 2\pi f$).

The physical interpretation is that a function can be decomposed into the sum
of a series of sine waves, with different amplitudes and phases at each
wavelength/frequency. \emph{Because we have amplitude and phase, the Fourier
transform is, in general, a complex function}.

Fourier transforms can be determined for discrete (as opposed to continuous)
streams of data using the \emph{discrete} Fourier transform, which replaces the
integrals with sums. Algorithms have been developed to compute the discrete
Fourier transform quickly by means of the \emph{Fast Fourier Transform (FFT)};
generally, this is done for data sets that are \textcolor{red}
{padded to a length of a power of 2 (???)}.
However, the FFT algorithm requires equally spaced data points. For
unequally spaced points, the full discrete Fourier transform is required.

Python implementations:
\href{http://docs.scipy.org/doc/numpy/reference/routines.fft.html}
{\tt numpy.fft},
\href{http://docs.scipy.org/doc/scipy/reference/tutorial/fftpack.html}
{\tt scipy.fftpack}

\begin{verbatim}
import matplotlib.pyplot as plt
import numpy as np

# generate sine wave
x = np.linspace(0, 10000., 8192)
y = np.sin(100*x)
plt.plot(x,y)

# Do a fast fourier transform
f = np.fft.fft(y)

# plot amplitude, try to get frequencies right...
plt.plot(np.abs(f)) \\
plt.plot(np.fft.fftfreq(8192), np.abs(f))
plt.plot(np.fft.fftfreq(8192,100./8192), np.abs(f))
\end{verbatim}

Two particular operations involving pairs of functions are of general
interest: convolution and cross-correlation.

\begin{itemize}[itemsep=1ex]
    \item {\bf Convolution}$$ g(x)*h(x) = \int\!g(x')h(x-x')\mathrm{d}x'$$
        Usually, convolution is seen in the context of
        \emph{smoothing}, where $h(x)$ is a normalized function (with a sum
        of unity), centered on zero. Convolution is the process of running
        this function across an \emph{input function} to produce a smoothed
        version. Note that the process of convolution is computationally
        expensive; at each point in a data series, you have to loop over
        all of the points (or at least those that contribute to the
        convolution integrat [integrand? integral?]).
        In some cases, it can be advantageous to consider convolution in
        the Fourier domain because of the \emph{convolution
        theorem}, which states that convolving two functions in physical
        space is equivalent to multiplying the transforms of the functions
        in Fourier space. Multiplication of two functions is
        computationally faster than convolution).

        \href{http://docs.scipy.org/doc/numpy/reference/generated/numpy.convolve.html}
        {\tt numpy.convolve}

    \item {\bf Cross-correlation}
        $$ g(x)\star h(x)=\int\! g(x')h(x+x')\mathrm{d}x'$$
        Cross-correlation is actually the same as convolution if $h$
        is a symmetric function, but is usually used quite differently.
        It is generally considered a function of the \emph{lag}, $x$.
        Multiply two functions, calculate the sum, then shift one of the
        functions and do it again. Then look at the sums as a function of
        the shift.
        For two similar functions, the cross-correlation will be a maximum
        when the two functions ``line up'', so
        \textcolor{red}{this is useful for
        determining shifts between two functions, e.g.\ spatial shifts of
        images, or spectral shifts from velocity (but in velocity space,
        i.e.\ log(wavelength), not linear wavelength space).}
        Cross-correlation can also be computed in the Fourier domain. It is
        equivalent to multiplying the Fourier transform of one function by
        the complex conjugate of the Fourier transform of the other. Note
        that the height and width of the cross-correlation function has
        information about the degree to which the two functions are
        similar. A specific case of cross-correlation is the
        \emph{autocorrelation function}, which is the cross-correlation of
        a function with itself.
\end{itemize}

\test{Understand the basics of Fourier transforms. Understand what
convolution and cross-correlation are, and understand the convolution
theorem.}

Back to interpolation$\ldots$ See Bracewell chapter 10 for sinc
interpolation and the sampling theorem.

When we sample a function at finite locations, we are multiplying by a
\emph{sampling function}, a discrete set of impulses spaced by
$\Delta{x}$. In the Fourier domain, this is \emph{convolving} by the Fourier
transform of the sampling function, a discrete set of frequencies
spaced by $1/\Delta{x}$.

An important and useful result for interpolation is called the
\emph{sampling theorem} which arises in the case of a \emph{band-limited
function}, a function that has no power above some critical frequency
known as the \emph{Nyquist frequency}. In this case, if the function is
sampled at sufficient frequency, (twice the Nyquist frequency), it is
possible to recover the entire function. Figure 10.3 from Bracewell shows
the idea graphically.

What to do:
\begin{enumerate*}
    \item Fourier transform the sampled function.
    \item Multiply by a box function.
    \item Fourier transform back.
\end{enumerate*}
Alternatively, convolve
the sampled function with the Fourier transform of a box function,
which is the sinc function:
$$ \textrm{sinc}(x) = \frac{\sin(x)}{x} $$
This leads to \emph{sinc interpolation}, which is the ideal interpolation
in the case of a band-limited function. As mentioned at the beginning of
this section (out of place), this is one application of Fouier
decomposition, which is useful when considering a series of data points.

\test{Understand the sampling theorem, and be able to explain it by
reference to how the Fourier transform of a band-limited function that is
sufficiently sampled compares to the transform of one that is undersampled.}

{\bf Interpolation and Uncertainties:}

Generally all interpolation schemes can be considered as a sum of data
values multiplied by a set of interpolation coefficients at each
point:
$$ y(x) = \sum_{i=0}^{n} y_{i}P_{i}(x) $$
So far, we have been considering interpolation in perfect data.
But if there are uncertainties on the data points, there will be
uncertainties in the interpolated values.
These uncertainties can be derived by error propagation (if the errors on
the individual data points are uncorrelated):
$$ \sigma(y(x))^{2} = \sum\sigma(y(x_{i}))^{2}P_{i}(x)^{2} $$
However, the interpolated errors are now themselves correlated,
so using them
properly requires the \emph{full covariance matrix}.
$$ \sigma_{x}^{2}=\sigma_{u}^{2}\left(\frac{\partial{x}}{\partial{u}}\right)^{2} +
\sigma_{v}^{2}\left(\frac{\partial{x}}{\partial{v}}\right)^{2} +
2\sigma_{uv}^{2}\frac{\partial{x}}{\partial{u}}\frac{\partial{x}}{\partial{v}} $$
$$ \sigma_{uv}^{2} = \lim_{n\rightarrow\infty}\frac{1}{N}
\sum(u_{i}-<u>)(v_{i}-<v>)$$
In the presence of noise (uncertainties), it is often advantageous to
\emph{model} the data by fitting an underlying function to a tabulated set of
data points with some merit function criteria for determing what
function provides the best fit. The fit can then be used to predict
values at arbitrary location.

Be aware of mixing and matching interpolation and fitting: doing fits on
data sets with uncorrelated errors is substantially easier than doing
them on data sets with correlated errors.

\subsection{Differentiation and integration}
\begin{itemize*}
    \item Differentiation: finite differences
    \item Integration: simple
        rules, extended rules, reaching a desired accuracy.
    \item Monte Carlo integration
\end{itemize*}

\subsection{Differential equations}
\begin{itemize*}
    \item Runge-Kutta
    \item boundary value problems.
\end{itemize*}

% Section 8
\newpage
\section{Fitting}
\subsection{Overview: frequentism vs Bayesian}
Given a set of observations/data, one often wants to summarize and get at
underlying physics by fitting some sort of model to the data. The model
might be an empirical model or it might be motivated by some underlying
theory. In many cases, the model is parametric: there is some number of
parameters that specify a particular model out of a given class.

The general scheme for doing this is to define some merit function that is
used to determine the quality of a particular model fit, and choose the
parameters that provide the best match, e.g.\ a minimum deviation between
model and data, or a maximum probability that a given set of parameters
matches the data.

It is also important to understand how reliable
the derived parameters are.
The extent to which the model is consistent with your understanding of
uncertainties on the data is an indication of how good the model fit
actually is.

There are two different ``schools'' about model fitting: Bayesian and
frequentism.

\begin{description}[itemsep=0ex]
    \item [Frequentism] (also called the classical approach).
        Consider how \emph{frequently} a data set might be observed given
        some underlying model. {\bf $P(D|M)$} -- \emph{probability of
        observing a data set given a model}. The model that
        produces the observed data most frequently is viewed as the correct
        underlying model, and as such, gives the best parameters, along
        with some estimate of their uncertainty.
    \item [Bayesian]
        {\bf $P(M|D)$} -- \emph{probability that a model is correct
        given a data set}.
        It allows for the possibility that external information may prefer
        one model over another, and this is incorporated into the analysis
        as a \emph{prior}. It considers the probability of different models,
        and hence, the probability distribution functions of parameters.
        Examples of priors: fitting a Hess diagram with a combination of
        SSPs, with external constraints on allowed ages; fitting UTR data
        for low count rates in the presence of readout noise.
\end{description}
The frequentist paradigm has been mostly used in astronomy up until fairly
recently, but the Bayesian paradigm has become increasingly widespread. In
many cases they can give the same result, but with somewhat different
interpretation. In some cases, results can differ. The basic underpinning of
Bayesian analysis comes from {\bf Bayes theorem} of probabilities:
    $$ P(A|B) = \frac{P(B|A)P(A)}{P(B)} $$
where $P(A|B)$ is the conditional probability, i.e.\ the probability of $A$ given
that $B$ has occurred. Imagine there is some joint probability distribution
function of two variables, $p(x,y)$
(see \href{http://astronomy.nmsu.edu/holtz/a575/images/ML3.2.png}
{ML 3.2}; this could be a distribution
of stars as a function of effective temperature and luminosity). Think about
slices in each direction to get to the probability at a given point, and we
have:
    $$ p(x,y) = p(x|y)p(y) = p(y|x)p(x) $$
This gives:
    $$ p(x|y) = \frac{p(y|x)p(x)}{p(y)} $$
which is Bayes theorem. Bayes theorem also relates the conditional probability
distribution to the \emph{marginal} probability distribution:

\begin{align*}
    p(x) &= \int \! p(x|y)p(y) \ \mathrm{d}y \\
    p(y) &= \int \! p(y|x)p(x) \ \mathrm{d}y \\
    p(x|y) &= \frac{p(y|x)p(x)}{\int \! p(y|x)p(x) \ \mathrm{d}x}
\end{align*}
This is all fairly standard statistics, but the Bayesian paradigm extends
this to the idea that the quantity $A$ or $B$ can also represent a hypothesis,
or model, i.e.\ the relative probability of different models.

The context of Bayesian analysis involves the probability of models
(instead of data), and we have:
    $$ P(M|D) = \frac{P(D|M)P(M)}{P(D)} $$
In this case, P(D) is a normalization constant, P(M) is the prior on the
model, which, in the absense of any additional information, is equivalent
for all models: a \emph{noninformative prior}. (However, a noninformative
prior can itself be a prior; a uniform prior is not necessarily invarient
to a change in variables. For example, a uniform prior in the logarithm
of a variable is not uniform in the variable iteself).
In the noninformative case, the Bayesian
result is the same as the frequentist maximum likelihood. However, in the
Bayesian analysis we'd want to calculate the full probability distribution
function for the model parameter, $\mu$. More on that later
\textcolor{red}{(insert section reference here!)}

In practice, frequentist analysis yields parameters and their
uncertainties, while Bayesian analysis yields probability distribution
functions of parameters. The latter is often more computationally
intensive to calculate. Bayesian analysis includes explicit priors.

For some more detailed discussion, see
\url{http://arxiv.org/abs/1411.5018}

\test{Understand the basic conceptual differences between a frequentist
and a Bayesian analysis. Know Bayes' theorem. Understand the practical
differences.}

Starting with a frequentist analysis (which often provides a component of
the Bayesian analysis): given a set of data and some model (with
parameters), consider the probability that the data will be observed if the
model is correct, and choose the set of parameters that maximizes this
probability.

For example, when fitting a straight line through the data, the best fit is
determined by the method of \emph{least squares}. This comes from the
consideration that
observed data points have some measurement uncertainty,
$\sigma$ (for now, all points have homoschedastic (equal) uncertainties
that are distributed according to a Gaussian). The questions to ask are:

\begin{enumerate}
    \item What is the probability is of observing a given data value from a
        known Gaussian distribution?
    \item What is the probability of observing a
        series of two independently drawn data values?
\end{enumerate}

Suppose we are fitting $N$ data points $(x_i,y_i)$, where $i=0,\ldots,N-1$
to a model
that has $M$ adjustable parameters $a_j$, where $j=0,\ldots,M-1$.
The model predicts a functional relationship between the measured
independent and dependent variables:
$$ y(x) = y(x|a_0 \ldots a_{M-1}) $$
where the notation indicates dependence on the parameters explicitly on the
right-hand side, following the vertical bar.

So given some model $y(x_i|a_j)$, the probability of observing a
series of data points is:{$$
    P(D|M) \propto \prod^{N-1}_{i=0}\left\{\exp
    \left[-\frac{1}{2}\left[\frac{y_i-y(x_i|a_j)}
    {\sigma}\right]^{2}\right]\Delta y \right\}
$$}where $a_j$ is the $jth$ parameter of the model, and $\Delta y$ is some small
range in y.

Maximizing the probability is the same thing as
maximizing the \emph{logarithm} of the probability, which is the same thing as
\emph{minimizing} the \emph{negative} of the (natural?) logarithm
($-\log{x} = \log{x^{-1}} = \log(1/x)$).
In other words, minimize:{$$
    -\log P(D|M) = \sum^{N-1}_{i=0}\left\{\frac{1}{2}\left[
    \frac{y_i-y(x_i)}{\sigma}\right]^2\right\}
    -N\log(\Delta y) + const
$$}
If $\sigma$ is the same for all points, then this is equivalent to minimizing
$$ \sum^{N-1}_{i=1}\left[y_i-y\left(x_{i}|a_{j}\right)\right]^{2} $$
which is the least squares fit.

\test{Understand how least squares minimization can be derived from a maximum
    likelihood consideration in the case of normally-distributed
    uncertainties.}

Consider a simple application, where we have multiple measurements of some
quantity, so the model is $y(x_{i}) = \mu$, and we want to determine the
most probable value of $\mu$. To minimize $-\log{P(D|M)}$, take the
derivative and set it equal to zero. This gives:
\begin{align*}
    \frac{\mathrm{d}(-\log P(D|M))}{\mathrm{d}\mu}
    &= 2\sum^{N-1}_{i=0}(y_i-\mu) = 0 \\[1ex]
    \mu &= \sum \frac{y_i}{N}
\end{align*}
which should look familiar ($\mu$ = mean).
To calculate the uncertainty on the mean,
use error propagation:
$$  \sigma(\mu)^2 = \sigma_{y_0}^2
    \left( \frac{\partial{\mu}}{\partial{y}_0}\right)^2
    + \sigma_{y_1}^2
    \left( \frac{\partial{\mu}}{\partial{y}_1}\right)^2
    + \ldots $$
$$  \sigma_{\mu}^{2} = \sum\frac{\sigma^{2}}{N^{2}} = \frac{\sigma^{2}}{N}$$
$$  \sigma_{\mu} = \frac{\sigma}{\sqrt{N}} $$

\hrule
Side note: Error propagation
\begin{align*}
    x &= f(u,v)\\
    \sigma_{x}^{2} &=
    \sigma_{u}^{2}\left(\frac{\partial{x}}{\partial{u}}\right)^{2} +
    \sigma_{v}^{2}\left(\frac{\partial{x}}{\partial{v}}\right)^{2}
\end{align*}
\hrule

For \emph{heteroschedastic} (unequal) uncertainties on the data points:
$$ -\log P(D|M) = \sum_{i=0}^{N-1}\left[
\frac{(y_i-y(x_i))^{2}}{\sigma_{i}^{2}}\right] + const\
\textcolor{red}{\equiv \chi^2} $$

Here, we are minimizing a quantity called \textcolor{red}
{$\chi^{2}$}. An important thing about $\chi^{2}$ is that the probability
of a given value of $\chi^{2}$ given $N$ data points and $M$ parameters
can be calculated \emph{analytically}. This is called the
{\it $\chi^2$ distribution for $\nu = N-M$ degrees of freedom}.
(See
\href{http://docs.scipy.org/doc/scipy-0.15.1/reference/generated/scipy.stats.chi2.html}
{scipy.stats.chi2}. The cumulative density function (cdf), for example,
needs to be in some range, like 0.05 \- 0.95).

The quality of a fit can be qualitatively judged by the
\emph{reduced} $\chi^2$, or $\chi^2$ per degree of freedom:
$$  \chi_{\nu}^{2} = \frac{\chi^{2}}{N-M} = \frac{\chi^{2}}{\nu}
$$
which is expected to have a value near unity.
However, it is the probability of $\chi^2$ that should be calculated, not
$\chi^{2}$ itself since
the spread in $\chi_{\nu}^2$ depends on $\nu$ (the standard deviation
is $\sqrt{2\nu}$).

\textcolor{red}{It is important to recognize that this analysis depends
on the assumption that the uncertainties are distributed according to
a normal (Gaussian) distribution.}

$\chi^{2}$ can be used as a method for checking uncertainties, or even
determining them (at the expense of being able to say whether your
model is a good representation of the data).
See \href{http://astronomy.nmsu.edu/holtz/a575/images/ML4.1.png}
{ML 4.1}.

\test{Know specifically what $\chi^{2}$ is and how to use it.  Know what reduced
$\chi^{2}$ is, and understand degrees of freedom.}

As with linear least squares, to minimize $\chi^{2}$, take the derivative
with respect to the parameters, and set it equal to zero.
For our simple model of measuring a \emph{single quantity}, we have:
$$ \frac{\mathrm{d}(-\log P(D|M))}{\mathrm{d}\mu} =
    2\sum\frac{(y_{i}-\mu)}{\sigma_i^2} = 0
$$
$$ \sum\frac{y_i}{\sigma_i^2} - \mu\sum\frac{1}{\sigma_i^2} = 0
$$
$$ \mu = \frac{\sum\frac{y_i}{\sigma_i^2}}{\sum\frac{1}{\sigma_i^2}}
$$
i.e., a weighted mean. Again, you can use error
propagation to get (work not shown):
$$ \sigma_{\mu} = \left(\sum\frac{1}{\sigma_i^2}\right)^{-1/2}
$$
A more complicated example: fitting a straight line to a set of data.
Here, the model is
$$
    y(x_i) = a + bx_i
$$
and $\chi^2$ is:
\begin{align*}
    \chi^2 &= \sum_{i=0}^{N}\left[\frac{\left[y_i-(a+bx_i)\right]^2}{\sigma_i^2}\right]\\
           &= \sum_{i=0}^{N}\left[\frac{\left[y_i-a-bx_i\right]^2}{\sigma_i^2}\right]
\end{align*}
Again, we want to minimize $\chi^2$, so we take the derivatives with respect
to each of the parameters and set them to zero:
$$  \frac{\partial\chi^{2}}{\partial{a}} =
    -2\sum\frac{(y_i-a-bx_i)}{\sigma_i^2} = 0 $$
$$  \frac{\partial\chi^{2}}{\partial{b}} =
    -2\sum\frac{x_i(y_i-a-bx_i)}{\sigma_i^2} = 0 $$

Separating out the sums, we have:
$$  \sum\frac{y_i}{\sigma_i^2} -
   a\sum\frac{1}{\sigma_i^2} -
    b\sum\frac{x_i}{\sigma_i^2} = 0 $$
$$  \sum\frac{x_iy_i}{\sigma_i^2} -
    a\sum\frac{x_i}{\sigma_i^2} -
    b\sum\frac{x_i^2}{\sigma_i^2} = 0 $$
or
    $$ aS + bS_{x} = S_{y} $$
    $$ aS_{x} + bS_{xx} = S_{xy} $$
where the various $S$ are a shorthand for the sums:
    $$ S = \sum\frac{1}{\sigma_i^2}  $$
    $$ S_{x} = \sum\frac{x_i}{\sigma_i^2}  $$
    $$ S_{xy} = \sum\frac{x_iy_i}{\sigma_i^2}  $$
    $$ S_{xx} = \sum\frac{x_iy_i}{\sigma_i^2}  $$
    $$ S_{y} = \sum\frac{y_i}{\sigma_i^2}  $$
which is a set of two equations with two unknowns. Solving, you get:
$$ a = \frac{S_{xx}S_{y} - S_{x}S_{xy}}{SS_{xx} - S_{x}^{2}}  $$
$$ b = \frac{SS_{xy} - S_{x}S_{y}}{SS_{xx} - S_{x}^{2}}  $$

\test{Know how to derive the least squares solution for a fit to a straight
line.}

We also want the uncertainties in the parameters.
Again, use propagation of errors to get (work not shown):
$$ \sigma_a^2 = \frac{S_{xx}}{SS_{xx}-S_x^2}   $$
$$ \sigma_b^2 = \frac{S}{SS_{xx}-S_x^2}   $$

Finally, we will want to calculate the probability of getting
$ \chi^{2}$ for our fit,
in an effort to understand
\begin{enumerate}
    \item if our uncertainties are not properly calculated and
        normally distributed
    \item if our model is a poor model
\end{enumerate}

Let's do it. Simulate a data set, fit a line, and output parameters, parameter
uncertainties, $\chi^2$, and probability of $\chi^2$.

\subsection{General linear fits}
We can generalize the least squares idea to any model that is some linear
combination of terms that are a function of the independent variable,
e.g.
$$ y = a_0 + a_1f_1(x) + a_2f_2(x) + a_3f_3(x) + \ldots
$$
such a model is called a \emph{linear} model because it is linear
in the \emph{parameters}, $a_{i}$ but not necessarily linear in the independent
variable, $x$. The model could be a polynomial of arbitrary order, but
could also include trigonometric functions, etc. We write the model
in simple form:
$$  y(x) = \sum_{k=0}^{M-1}a_kX_k(x)
$$
where there are $M$ parameters, $N$ data points, and $N>M$.
The $\chi^2$ \textcolor{red}{merit function (???)} can be written as:
$$ \chi^2 = \sum_{i=0}^{N-1} \left[
    \frac{(y_i - \sum_{k=0}^{M-1} a_{k}X_{k}(x_{i}))^{2}}
    {\sigma_{i}^{2}} \right]
$$
Minimizing $\chi^2$ leads to the set of $M$ equations:
$$ 0 = \sum_{i=1}^{N-1}\frac{1}{\sigma_i^2}
    \left[y_{i} - \sum_{j=0}^{M-1}a_{j}X_{j}(x_{i})\right]X_{k}(x_{i})
$$
where $k=0,\ldots,M-1$.

Separating the terms and interchanging the order of the sums gives:
$$  \sum_{i=0}^{N-1}\frac{y_{i}X_{k}(x_{i})}{\sigma_{i}^{2}} =
    \sum_{j=0}^{M-1}\sum_{i=0}^{N-1}\frac{a_{j}X_{j}(x_{i})X_{k}(x_{i})}{\sigma_{i}^{2}}
$$
Define:
$$ \alpha_{jk} = \sum_{i=0}^{N-1}\frac{X_{j}(x_{i})X_{k}(x_{i})}{\sigma_{i}^{2}}
$$
$$ \beta_{j} = \sum_{i=0}^{N-1}\frac{X_{j}(x_{i})y_{i}}{\sigma_{i}^{2}}
$$
then we have the set of equations:
$$ \alpha_{jk}\alpha_{j} = \beta_{k} $$
for $k=0,\ldots,M-1$.

\test{Know the equations for a general linear least squares
problem, and how they are derived.}

Sometimes these equations are cast in terms of the \emph{design matrix},
$A$, which consists of $N$ measurements of $M$ terms:
$$ A_{ij} = \frac{X_j(x_i)}{\sigma_i}
$$
with $N$ rows and $M$ columns. Along with the definition:
$$ b_i = \frac{y_i}{\sigma_i} $$
we have:
$$ \alpha = A^{T} \cdot A $$
$$ \beta = A^{T} \cdot b $$
where the dots are for the matrix operation that produces the sums.
This is just another notation for the same thing, introduced here in
case you run across this language or formalism.

For a given data set, $\alpha$ and $\beta$ can be calculated either
by doing the sums or by setting up the disign matrix and using matrix
arithmetic. Then solve the set of equations for $\alpha_k$.

Note that this formulation applies to problems with multiple
independent variables, e.g., fitting a surface to a set of points;
simply treat $x$ as a vector of data, and the formulation is exactly
the same.

\subsection{Solving linear equations}

This is just a linear algebra problem, and there are well-developed techniques
(see NR chapter 2). For the most simple case, invert the matrix $\alpha$ to get
$$ \alpha_k = \alpha_{jk}^{-1}\beta_k
$$
A simple algorithm for inverting a matrix is called a \emph{Gauss-Jordan
elimination}. In particular, NR recommends the use of singular value
decomposition for solving all but the simplest least squares problems; this is
especially important if your problem is nearly singular, i.e.\ where two or
more of the equations may not be totally independent of each other: fully
singular problems should be recognized and redefined, but it is possible to
have non-singular problems encounter singularity under some conditions
depending on how the data are sampled. See NR 15.4.2 and chapter 2 (in these
notes?).

As an aside, note that it is possible to solve a linear set of equations
significantly faster than inverting the matrix through various types of
matrix decomposition, e.g., LU decomposition and Cholesky decomposition
(see NR Chapter 2). There may be linear algebra problems that you
encounter that only require the solution of the equations and not the
inverse matrix. However, for the least squares problem, we often
do want the inverse matrix $C = \alpha^{-1}$, because its elements
have meaning. Propagation of errors gives:
$$ \sigma^{2}(a_{k}) = C_{jj}
$$
i.e., the diagonal elements of the inverse matrix. The off-diagonal
elements give the covariances between the parameters, and the inverse
matrix, $C$, is called the \emph{covariance matrix}.

A Python implementation of matrix inversion and linear algebra in general
can be found in
\href{http://docs.scipy.org/doc/scipy/reference/tutorial/linalg.html}
{scipy.linalg}.
Here is the summed matrix implementation for a straight line
(but the functional form only comes in through the derivative function.):
\begin{verbatim}
    def deriv(x) :
        # if x is numpy array, then we can return vectors of derivatives
        try :
            return [np.ones(len(x)),x]
        except :
            return [1.,x]

    # data points in (x,y)
    # loop over parameters, sums over data points (y) are done with vector arithmetic
    for k in np.arange(npar) :
        beta[k] = np.sum(deriv(x)[k]*y/sigma**2)
        for j in np.arange(npar) :
            alpha[k,j] = np.sum(deriv(x)[k]*deriv(x)[j]/sigma**2)
    c=np.linalg.inv(alpha)
    print np.dot(c,beta)
\end{verbatim}
Here is the design matrix approach to the sums:
\begin{verbatim}
    A = np.vander(x, 2)  # Take a look at the documentation to see what this function does.
    ATA = np.dot(A.T, A / yerr[:, None]**2)
    w = np.linalg.solve(ATA, np.dot(A.T, y / yerr**2))
    V = np.linalg.inv(ATA)
\end{verbatim}
Linear least squares is also implemented in \texttt{astropy} in the
\href{http://astropy.readthedocs.org/en/v1.0.6/modeling/index.html}
{modeling module} using the
\href{http://astropy.readthedocs.org/en/v1.0.6/api/astropy.modeling.fitting.LinearLSQFitter.html#astropy.modeling.fitting.LinearLSQFitter}
{LinearLSQFitter} class (which uses {\tt numpy.linalg});
{\tt astropy} has a number of standard models to use,
or you can provide your own custom model.
\begin{verbatim}
    from astropy.modeling import models, fitting
    fit_p = fitting.LinearLSQLSQFitter()
    p_init = models.Polynomial1D(degree=degree)
    pfit = fit_p(p_init, x, data)
\end{verbatim}

{\tt astropy} has a number of common models, but you can also define your own
model using
\href{http://astronomy.nmsu.edu/holtz/a575/ay575notes/astropy.readthedocs.org/en/v1.0.6/api/astropy.modeling.custom_model.html#astropy.modeling.custom_model}
{models.custom\_model}, by supplying a function that returns
values, and a function that returns derivatives with respect to the
parameters.

\test{Be able to computationally implement a linear least squares solution
to a problem.}

Note that for problems with large numbers of parameters, the linear algebra
can become very computationally expensive. In many cases, however, the
matrix that represents these problems may only be sparsely populated,
i.e.\ so-called sparse matrices (see Numerical Recipes 2.7, Figure 2.7.1).

Example: spectral extraction

In such cases, there are methods that allow the equations to be solved
significantly more efficiently.

However, not all fits are linear in the parameters. Which of the
following are not?
\begin{itemize}
    \item $y = a_{0}x + a_{1}e^{-x}$
    \item $y = a_{0}(1 + a_{1}x)$
    \item $y = a_{0}\sin(x - a_{1})$
    \item $y = a_{0}\sin^{2}x$
    \item $y = a_{0}e^{(-(x-a_{1})^{2}/a_{2}^{2})}$
\end{itemize}

\subsection{Nonlinear fits}
In a linear fit, the $\chi^{2}$ surface is parabolic in parameter
space, with a single minimum, and linear least squares can be used to
determine the location of the minimum. In the non-linear case, however, the
$\chi^{2}$ surface can be considerably more complex, and there is a
(good) possibility that there are multiple minima, so one needs to be
concerned about finding the global minimum and not just a local minimum.
Because of the complexity of the surface, the best fit is found by an
iterative approach.

A conceptually simple approach would be a \emph{grid search}, where one simply
tries all combinations of parameters and finds the one with the lowest
$\chi^{2}$. Obviously, this is extremely computationally inefficient,
especially for problems with more than a few parameters. One is also forced
to decide on a step size in the grid, although you might imagine a
successively refined grid as you proceed. But in general, this method is
not recommended, apart from occaisionally trying it to try to ensure that
your more efficient solution is not landing in a local minimum.

Better approaches attempt to use the $ \chi^{2}_{}$ values to find the
minimum more efficiently. They can generally be split into two classes:
those that use the derivative of the function and those that don't. If you
can calculate derivatives of $ \chi^{2}_{}$ with respect to your
parameters, then this provides information about how far you can move in
parameter space towards the minimum. If you can't calculate derivates, you
can evaluate $ \chi^{2}_{}$ at several different locations, and use these
values to try to work your way towards the minimum.

With derivatives, the approach has a fairly close parallel to the linear
least squares problem. Around the final minimum, the $ \chi^{2}_{}$ surface
can be approximated as a parabola, and it is possible to correct the
solution to the minimum solution if one can arrive at a set of parameters
near to the final minimum. This is acheived via the set of equations:
$$
    \sum_{i=0}^{M-1}\alpha_{kl}\delta a_{1} = \beta_{k}
$$
where
$$
    \beta_{k} = -\frac{1}{2}\frac{\partial\chi^{2}}{\partial a_{k}} =
    \sum_{i=0}^{N-1}\frac{(y_{i}-y(x_{i}|a))}{\sigma_{i}^{2}}
    \frac{\partial(y(x_{i}|a))}{\partial a_{k}}
$$
$$
    \alpha_{kl} =
    \frac{1}{2}\frac{\partial^2\chi^2}{\partial a_k\partial a_l} =
    \sum_{i=0}^{N-1}\left[
        \frac{1}{\sigma_i^2}\frac{\partial y(x_i|a)}{\partial a_k}
        \frac{\partial y(x_i|a)}{\partial a_l} -
        (y_i-y(x_i|a))
        \frac{\partial^2y(x_i|a)}{\partial{a_k}\partial{a_l}}
    \right]
$$
The matrix $\alpha_{kl}$ is known as the \emph{curvature matrix}. In most
cases, it is advisable to drop the second derivative term in this matrix
(see NR 15.5.1 for a partial explanation). The standard approach uses:
$$
    \alpha_{kl} = \sum_{i=0}^{N-1}\left[\frac{1}{\sigma_i^2}
    \frac{\partial{y}(x_i|a)}{\partial{a_k}}
    \frac{\partial{y}(x_i|a)}{\partial{a_l}} \right]
$$
To implement this, you choose a starting guess of parameters, solve
for the corrections  $\delta a_l$ to be applied to a current set of
parameters, and iterates until one arrives at a solution that does not
change significantly.

Far from the solution, this method can be significantly off in
providing a good correction, and, in fact, can even move parameters
away from the correct solution. In this case, it may be advisable to
simply head in the the steepest downhill direction in $ \chi^{2}_{}$
space, which is known as the method of steepest descent. Note that
while this sounds like a reasonable thing to do in all cases, it can
be very inefficient in finding the final solution
(see, e.g.\ \href{http://astronomy.nmsu.edu/holtz/a575/images/NR10.8.1.png}
{NR Figure 10.8.1}).

A common algorithm switches between the method of steepest descent and
the parabolic approximation and is known as the Levenberg-Marquardt
method. This is done by using a modified curvature matrix:
$$ \sum_{i=0}^{M-1} \alpha_{kl}'\delta{a_1} = \beta_{k} $$
where
$$ \alpha_{jj}' = \alpha_{jj}(1+\lambda) $$
$$ \alpha_{jk}' = \alpha_{jk}\ (\textrm{for}\ j \neq k) $$

When $\lambda$ is large, this gives a \emph{steepest descent} method.
When $\lambda$ is small, this gives a \emph{parabolic} method.
This leads to the following recipe:
\begin{enumerate}
    \item Choose a starting guess of parameter vector ($a$) and
        calculate $\chi^2(a)$.
    \item Calculate the correction using model $\lambda$,
        e.g.\ $\lambda$ = 0.001 and evaluate $\chi^2$ at the new point.
    \item If $\chi^2(a + \delta a) > \chi^2(a)$, increase $\lambda$ and try again.
    \item If $\chi^2(a + \delta a) < \chi^2(a)$, decrease $\lambda$,
        update $a$, and start the next step.
    \item Stop when the convergence criteria are met.
    \item Invert the curvature matrix to get parameter uncertainties.
\end{enumerate}

Two key issues in nonlinear least squares is finding a good starting
guess and a good convergence criterion. You may want to consider
multiple starting guesses to verify that you're not converging to a
local minimum. For convergence, you can look at the amplitude of the
change in parameters or the amplitude in the change of $ \chi^{2}_{}$.

This method is implemented in \texttt{astropy} in the
\href{http://astropy.readthedocs.org/en/v1.0.6/modeling/index.html}
{modeling module} using the
\href{http://docs.astropy.org/en/stable/api/astropy.modeling.fitting.LevMarLSQFitter.html}
{LevMarLSQFitter} which is used identically to the linear least squares
fitter described above.

There are certainly other approaches to non-linear least squares, but
this provides an introduction.


\subsubsection{A nonlinear fitter without derivatives}
A common minimization method that does not use derivatives is the
``downhill simple'' or \emph{Nelder-Mead} algorithm.
For this technique, $\chi^{2}$ is initially evaluated at $M+1$ points
(where $M$ is the number of parameters).
This makes an $M$-dimensional figure, called a \emph{simplex}.
The largest value is found, and a trial evalaution is made at
a value reflected through the volume of the simplex. If this is
smaller than the next-highest value of the original simplex, a value
twice the distance is chosen and tested; if not, then a value half the
distance is tested, until a better value is found. This point replaces
the initial point, and the simplex has moved a step; if a better value
can't be found, the entire simplex is contracted around the best
point, and the process tries again. The process in then repeated until
a convergence criterion is met. See
\href{}
{NR Figure 10.5.1} and section 10.5.
The simplex works its way down the $ \chi^{2}_{}$ surface, expanding
when it can take larger steps, and contracting when it needs to take
smaller steps. Because of this behavior, it is also known as the
``amoeba'' algorithm.

\href{http://docs.scipy.org/doc/scipy-0.16.0/reference/generated/scipy.optimize.fmin.html}
{Python/scipy implementation of Nelder-Mead}

\href{https://docs.scipy.org/doc/scipy-0.16.1/reference/tutorial/optimize.html}
{Python/scipy summary of optimization routines in general}

\test{Understand the concepts of how a nonlinear least squares fit is
performed, and the issues with global vs local minima,
starting guesses, and convergence criteria. Know how to
computationally imiplement a nonlinear least squares solution.}

\subsection{Parameter uncertainties and confidence limits} Fitting provides us
with a set of best-fit parameters, but, because of uncertainties and limited
number of data points, these will not necessarily be the true parameters. One
generally wants to understand how different the derived parameters might be
from the true ones.

The covariance matrix gives information about the uncertainties on the derived
parameters. In the case of well-understood uncertainties that are strictly
distributed according to a Gaussian, these can be used to provide
\emph{confidence levels} on your parameters; see NR 15.6.5.

However, if the uncertainties are not well-understood, numerical techniques can
be used to derive parameter uncertainties.

A straightforward technique if you have a good understanding of your
\emph{measurement} uncertainties is the \emph{Monte Carlo simulation}. In this
case, you simulate your data set multiple times, derive parameters for each
simulated data set, and look at the range of fit parameters as compared with
the input parameters.  To be completely representative of your uncertainties,
you would need to draw the simulated data set from the true distribution, but
you don't know what that is (it's what you're trying to derive).  So we take
the best-fit from the actual data as representative of the true distribution,
and hope that the \emph{difference} between the derived parameters from the
simulated sets and the input parameters is representative of the difference
between the actual data set and the true parameters.

If you don't have a solid understanding of your \emph{data} uncertainties, then
\emph{Monte Carlo} will not give an accurate representation of your parameter
uncertainties. In this case, you can use multiple samples of your own data to
get some estimate of the parameter uncertainties. A common technique is the
\emph{bootstrap} technique, where, if you have $N$ data points, you make
multiple simulations using the same data, drawing $N$ data points from the
original set \- {\bf with replacement} (i.e.\ the same data point can be drawn
more than once), derive parameters from multiple simulations, and look at the
distribution of these parameters.

However, you may need to be careful about the interpretation of the confidence
intervals determined by any of these techniques, which are based on a
frequentist interpretation of the data. For these calculations, note that
confidence intervals will change for different data sets. The frequentist
interpretation is that the true value of the parameter will fall within the
confidence levels at the frequency specified by your confidence interval. If
you happen to have taken an unusual (\emph{in}frequent) data set, the true
parameters may not fall within the confidence levels derived from
this data set.

\test{Understand how uncertainties can be estimated from least-squares
fitting via the covariance matrix, Monte Carlo simulation, and the
bootstrap method, and how these techniques work}.

\subsection{Bayesian analysis}
Review material above, also see astroML, NR and
\url{http://arxiv.org/abs/1411.5018}

As discussed above and in additional references, Bayesian analysis has
several significant differences from the frequentist/likelihood
approach. First, Bayesian analysis calculates the probability of model
parameters given data, rather than the probability of data given a
model.

$$ P(M|D) = \frac{P(D|M)P(M)}{P(D)} $$

Practically, this allows for the possibility of specifying an explicit
prior on the model parameters. The Bayesian analysis may give the same
result as the frequentist analysis for some choices of the prior, but
it makes the prior explicit, rather than hidden.

Second, the Bayesian analysis produces a joint probability
distribution function of the model parameters. The frequentist
analysis produces a most probable set of parameters and a set of
confidence intervals.

The Bayesian analysis is computationally more challenging because you
actually have to compute the probability for different model
parameters. This is essentially performing a grid search of the full
parameter space, which can be computationally intractable for large
numbers of parameters. Fortunately, computational techniques have been
developed to search through the full parameter space concentrating
only on the regions with non-negligible probability; as discussed
below, Markov Chain Monte Carlo is foremost among such techniques.

\subsubsection{The prior}
Priors in Bayesian analysis reflect external knowledge about
parameters that exist independent of the data being analyzed. If there
is no external knowledge, then you want to use a \emph{noninformative} prior.
Often, this is just a statement that all parameter values are equally likely.
In many cases, this makes a Bayesian analysis equivalent to a frequentist
analysis. However, beware that model parameters could be expressed with a
transformation of variables, and a flat distribution in one variable may not be
a flat distribution under a variable transformation, and could lead to
different results. Also, a prior that appears to be unbiased may not be so,
e.g., consider the fitting parameters for a straight line, in particular the
\href{http://astronomy.nmsu.edu/holtz/a575/images/slope.png}
{slope}.

\subsubsection{Marginalization}
For many problems, only some of the parameters are of interest, while
others fall into the category of so-called \emph{nuisance} parameters, which
may be important for specifying an accurate model (which is
fundamentally important), even though they may not be of interest for
the scientific question you are trying to answer. In such cases,
Bayesian analysis uses the concept of \emph{marginalization}, where one
integrates over the dimensions of the nuisance parameters to provide
probability distribution functions of only the parameter(s) of
interest.

Marginalization is also used to give probability distribution
functions of individual parameters, i.e.\ without regard to the values
of other parameters. However, one needs to be aware that parameter
values may be correlated with each other, and marginalization hides
such correlations.

Marginalization can also be used to derive the probability
distribution of some desired quantity given measurements of other
quantities if there is some relation between the measured and desired
quantity. An example is in determining ages of stars given a set of
observations, e.g.\ of spectroscopic parameters and/or apparent
magnitudes. Here, stellar evolution gives predictions for observed
quantities as a function of age, mass, and metallicity; in conjunction
with an initial mass function, you get predictions for number of stars
for each set of parameters. Given some priors on age, mass, and/or
metallicity, you could compute the probability distribution of a given
quantity by marginalizing over all other parameters. Observational
constraints would modify this probability distribution.

\test{Understand what marginalization means, and why and when it can
be used}.

\subsubsection{Markov Chain Monte Carlo (MCMC)}
Calculating marginal probability distributions is basically a big
integration problem. If the problem has many parameters, the
multi-dimensional integral can be very intensive to calculate.
One technique for multi-dimensional integration is
{\bf Monte Carlo} integration. Choose a (large) number of points
at random within some specified volume (limits in multiple
dimensions), sample the value of your function at these points
and estimate the integral as

$$ \int\! f\mathrm{d}V = V<f> $$
where
$$ <f> = \frac{1}{N}\sum f(x_i) $$

We could use this to calculate marginal probability distribution functions, but
it is likely to be very inefficient if the probability is small over most of
the volume being integrated over. Also, for most Bayesian problems, we do not
have a proper probability density function because of an unknown normalizing
constant; all we have is the relative probability at different points in
parameter space.

To overcome these problems, we would instead like to place points in a volume
proportional to the probability distribution at that point; we can then
calculate the integral by summing up the number of points. This is achieved by
a {\bf Markov Chain Monte Carlo} analysis. Here, unlike Monte Carlo, the
points we choose are not statistically independent, but are chosen such that
they sample the (unnormalized) probability distribution function in proportion
to its value. This is acheived by setting up a Markov Chain, a process where
the value of a sampled point depends only on the value of the previous point.
To get the Markov Chain to sample the (unnormalized) PDF, the transistion
probability has to satisfy:

$$ \pi(x_{1})p(x_{2}|x_{1}) = \pi(x_{2})p(x_{1}|x_{2})  $$

where $p$ is a transition probability to go from one point to another.

To do the integral, sum the number of points in the chain at the different
parameter values.

There are multiple schemes to generate such a transition probability.  A common
one is called the \emph{Metropolis-Hastings} algorithm: from a given point,
generate a candidate step from some proposal distribution, $q(x2|x1)$, that is
broad enough to move around in the probability distribution $q(x2|x1)$ in steps
that are not too large or too small. If the candidate step falls at a larger
value of the probability function, accept it, and start again. If it falls at a
smaller value, accept it with probability proportional to

$$ \pi(x_c)q(x_1|x_{2c}) \pi(x_1)q(x_{2c}|x_1)  $$

For a symmetric Gaussian proposal definition, note that the $q$ values
cancel out.

Note also that you need to choose a starting point. If this is far from the
maximum of the probability distribution, it may take some time to get to the
equilibrium distribution. This leads to the requirement of a \emph{burn-in}
period for MCMC algorithms.

In practice, you have to be careful about choosing an appropriate
proposal distribution, and an appropriate burn-in period.

Implementation: see \href{http://dan.iel.fm/emcee/current/}
{emcee} and \href{https://pymc-devs.github.io/pymc/}
{pymc}, among others. Also
\href{http://jakevdp.github.io/blog/2014/06/14/frequentism-and-bayesianism-4-bayesian-in-python/}
{this}.

Note that {\tt emcee} uses an algorithm that is different from
Metropolis-Hastings, that is designed to function more efficiently in a case
where the PDF may be very narrow, leading to very low acceptance probabilities
with M-H. Partly this is accomplished by using a number of \emph{walkers},
i.e.\ multiple different Markov chains, but these are not totally independent:
knowledge from other walkers are used (somehow) to help explore the parameter
space more efficiently. Note that since the walkers are not independent, the
chain from a single walker does not sample the PDF proportionally; the combined
chain from all of the walkers is what needs to be used.

\test{Understand what Markov Chain Monte Carlo analysis accomplishes, and at
least qualitatively how it works. Be able to implement an MCMC analysis of
a problem.}

\end{document}
